\chapter{基础准备}
\begin{introduction}
	\item 点
	\item 圆
	\item 矩形
	\item 抛物线
	\item 文本
	\item 直线
	\item 抛物线
	\item pgfplots
\end{introduction}
\section{点(Point Path)}
一行的中间高度画一个半径为一半行高的红点\tikz \filldraw [red ] (0.5 ex ,0.5ex ) circle [radius =0.5 ex ];使用coordinate命令或者path命令附带coordinate来定义一个点。
\begin{lstlisting}
\begin{tikzpicture}
\draw[step=1,color=gray!40] (-2,-2) grid (2,2);
\path (1,1) coordinate (p1);
\coordinate (p2) at ( 2, 0);
\draw[dotted, red] (p1) -- (p2)  ;
\end{tikzpicture}
\end{lstlisting}
\begin{center}
\begin{tikzpicture}
\draw[step=1,color=gray!40] (-2,-2) grid (2,2);
\path (1,1) coordinate (p1);
\coordinate (p2) at ( 2, 0);
\draw[dotted, red] (p1) -- (p2)  ;
\end{tikzpicture}
\end{center}
\subsection{控制点}
起点$x$的控制点 $y$,指的是曲线所在点 $x$ 处的切线方向指向 $y$ 点。如图所示。点 $x_{1} (0,0)$ 处的切线方向指向点$ y_{1} (1,1)$,点 $x_{2} (2,0)$ 点的切线方向指向点 $y_{2} (2,1)$,用法如下:
\begin{verbatim}
\draw[options] (x1,y1) .. controls (x2,y2) and (x3,y3) .. (x4,y4);
\end{verbatim}
\begin{lstlisting}
\begin{tikzpicture}
\filldraw[gray] (0,0) circle (2pt) (1,1) circle (2pt)
(2,1) circle (2pt) (2,0) circle (2pt);
\draw (0,0) .. controls (1,1) and (2,1) .. (2,0);
\end{tikzpicture}
\begin{tikzpicture}
\filldraw (0,0) circle [radius =2 pt ] node [left ] {$ x _1$};
\filldraw [gray ] (1,1) circle [radius =2 pt ] node [left ] {$ y _1$};
\filldraw [gray ] (2,1) circle [radius =2 pt ] node [right ] {$ y_2$};
\filldraw (2,0) circle [radius =2 pt ] node [right ] {$ x _2$};
\draw (0,0) .. controls (1,1) and (2,1) .. (2,0); % 核心代码
\end{tikzpicture}
\begin{tikzpicture}
\draw (0,0) .. controls (1,1) and (2,1) .. (4,0);
\fill (1,1) circle (1pt) (2,1) circle (1pt);
\end{tikzpicture}
\end{lstlisting}
\begin{center}
	\begin{tikzpicture}
	\filldraw[gray] (0,0) circle (2pt) (1,1) circle (2pt)
	(2,1) circle (2pt) (2,0) circle (2pt);
	\draw (0,0) .. controls (1,1) and (2,1) .. (2,0);
	\end{tikzpicture}
\begin{tikzpicture}
\filldraw (0,0) circle [radius =2 pt ] node [left ] {$ x _1$};
\filldraw [gray ] (1,1) circle [radius =2 pt ] node [left ] {$ y _1$};
\filldraw [gray ] (2,1) circle [radius =2 pt ] node [right ] {$ y_2$};
\filldraw (2,0) circle [radius =2 pt ] node [right ] {$ x _2$};
\draw (0,0) .. controls (1,1) and (2,1) .. (2,0); % 核心代码
\end{tikzpicture}
\begin{tikzpicture}
\draw (0,0) .. controls (1,1) and (2,1) .. (4,0);
\fill (1,1) circle (1pt) (2,1) circle (1pt);
\end{tikzpicture}
\end{center}
再来一个控制点的例子:用控制点画一个半圆,见下图。当然,本例只是阐释控制点,实际中很少用这种方式画半圆。
\begin{lstlisting}
\begin{tikzpicture}
\draw (- 1.5,0) -- (1.5,0);
\draw (0,- 0.5) -- (0,1.5);
% show the control points
\filldraw [gray ] ( - 1,0) circle [radius =2 pt ]
(1,0) circle [radius =2 pt ]
[cyan ] (- 1,0.555) circle [radius =2 pt ]
[cyan ] (- 0.555,1) circle [radius =2 pt ]
[cyan ] (0.555,1) circle [radius =2 pt ]
[cyan ] (1,0.555) circle [radius =2 pt ];
\draw (- 1,0) .. controls (- 1,0.555) and (- 0.555,1) .. (0,1) .. controls (0.555,1) and (1,0.555) .. (1,0);
\end{tikzpicture}
\end{lstlisting}
\begin{center}
\begin{tikzpicture}
\draw (- 1.5,0) -- (1.5,0);
\draw (0,- 0.5) -- (0,1.5);
% show the control points
\filldraw [gray ] ( - 1,0) circle [radius =2 pt ]
(1,0) circle [radius =2 pt ]
[cyan ] (- 1,0.555) circle [radius =2 pt ]
[cyan ] (- 0.555,1) circle [radius =2 pt ]
[cyan ] (0.555,1) circle [radius =2 pt ]
[cyan ] (1,0.555) circle [radius =2 pt ];
\draw (- 1,0) .. controls (- 1,0.555) and (- 0.555,1) .. (0,1) .. controls (0.555,1) and (1,0.555) .. (1,0);
\end{tikzpicture}
\end{center}
贝塞尔曲线是四个点画出一个曲线,。其中第一个点是起点,第四个点终点,然后另外两个点是控制点。
\begin{lstlisting}
\begin{tikzpicture}[scale=3] 
\draw[help lines] (0,0) grid (2,2); 
\draw[color=red] (0,0) .. controls (1,1) and (2,1) .. (2,0); 
\shade[ball color=gray!10] (0,0) circle (0.1); 
\shade[ball color=gray!40] (1,1) circle (0.1); 
\shade[ball color=gray!70] (2,1) circle (0.1); 
\shade[ball color=gray] (2,0) circle (0.1); 
\end{tikzpicture}
\end{lstlisting}
\begin{center}
\begin{tikzpicture}[scale=3] 
\draw[help lines] (0,0) grid (2,2); 
\draw[color=red] (0,0) .. controls (1,1) and (2,1) .. (2,0); 
\shade[ball color=gray!10] (0,0) circle (0.1); 
\shade[ball color=gray!40] (1,1) circle (0.1); 
\shade[ball color=gray!70] (2,1) circle (0.1); 
\shade[ball color=gray] (2,0) circle (0.1); 
\end{tikzpicture}
\end{center}
\subsection{点的相对偏移}
tikz中有一个重要的概念,当前点,然后点可以通过当前点根据相对偏移来确定一个新的点。上面代码第9行的 ++ 符号和第10行的 + 符号都根据当前点然后进行了 $\Delta x$ 和 $\Delta y$ 的相对偏移从而确定了一个新的点。这两个符号的区别在于是不是更新当前点数据。++符号更新当前点,而+符号不更新。

++适合描述一连串逐渐变化的点,+适合描述多个点围绕着一个点变化的情况。
\begin{lstlisting}
\begin{tikzpicture}[scale=1]
\draw[step=1,color=gray!40] (-2,-2) grid (2,2);
\draw[latex-latex, red] (0,-2) -- ++(-1,1) -- ++(-1,-1);
\draw[dashed, blue] (0,1) -- +(-1,1) -- +(-2,0);
\end{tikzpicture}
\end{lstlisting}
\begin{center}
	\begin{tikzpicture}[scale=1]
	\draw[step=1,color=gray!40] (-2,-2) grid (2,2);
	\draw[latex-latex, red] (0,-2) -- ++(-1,1) -- ++(-1,-1);
	\draw[dashed, blue] (0,1) -- +(-1,1) -- +(-2,0);
	\end{tikzpicture}
\end{center}

\subsection{node命令中点的定义}
tikz中的点也支持极坐标表示,(30:1cm),第一个参数是极座标里面的角度,第二个参数是半径。
\begin{lstlisting}
\begin{tikzpicture} 
\node (node001) at (0,2) [draw] {test math competition}; 
\end{tikzpicture}
\end{lstlisting}
\begin{center}
	\begin{tikzpicture} 
	\node (node001) at (0,2) [draw] {test math competition}; 
	\end{tikzpicture}
\end{center}

从这里可以看到只要写上draw选项外面就会加上一个长方形,也就是shape的默认选项是rectangle。如果你不希望外面有长方形,不写draw选项即可。

这里通过node命令定义了一个点,node001,在(0,2)那里。后面是可以使用的。
\begin{lstlisting}
\begin{tikzpicture} 
\node (node001) at (0,2) [draw] {node001}; 
\node (node002) at (-2,0) [draw] {node002}; 
\node (node003) at (2,0) [draw] {node003}; 
\draw (node cs:name=node003,anchor=north) |- (0,1); 
\draw (node002.north) |- (0,1) -| (node cs:name=node001,anchor=south); 
\end{tikzpicture}
\end{lstlisting}
\begin{center}
	\begin{tikzpicture} 
	\node (node001) at (0,2) [draw] {node001}; 
	\node (node002) at (-2,0) [draw] {node002}; 
	\node (node003) at (2,0) [draw] {node003}; 
	\draw (node cs:name=node003,anchor=north) |- (0,1); 
	\draw (node002.north) |- (0,1) -| (node cs:name=node001,anchor=south); 
	\end{tikzpicture}
	
\end{center}

这里通过 node cs:name=node003 来获取之前那个node所在的点,然后通过 anchor=north 来定义那个node的接口在北边。除此之外的选项还有: south ,east ,west 。这里 |- 似乎是画垂直拐线的意思。上面的语法简写为可以node002.north。

此外还有 angle 选项控制node接口的开口角度。

\subsection{两个点定义出一个点}
\begin{lstlisting}
\begin{tikzpicture} 
\node (p1) at (30:1) {$p_1$} ; 
\node (p2) at (75:1) {$p_2$} ; 
\draw (-0.2,0) -- (1.2,0) node[right] (xline) {$q_1$}; 
\draw (2,-0.2) -- (2,1.2) node[above] (yline) {$q_2$}; 
\draw[->] (p1) -- (p1 |- xline); 
\end{tikzpicture}

\end{lstlisting}
\begin{center}
	\begin{tikzpicture} 
	\node (p1) at (30:1) {$p_1$} ; 
	\node (p2) at (75:1) {$p_2$} ; 
	\draw (-0.2,0) -- (1.2,0) node[right] (xline) {$q_1$}; 
	\draw (2,-0.2) -- (2,1.2) node[above] (yline) {$q_2$}; 
	\draw[->] (p1) -- (p1 |- xline); 
	\end{tikzpicture}
\end{center}

这种形式 (p1 |- xline) 表示取第一个点的$x$和第二个点的$y$组成一个新的点。如果是 (p1 -| xline) 表示取第二个点的$x$和第一个点的$y$组成一个新的点。

两个path的交点
\begin{lstlisting}
\begin{tikzpicture}[scale=3] 
\draw[help lines] (0,0) grid (2,2); 
\coordinate (A) at (0,0); 
\coordinate (B) at (2,0.5); 
\coordinate (C) at (2,0); 
\coordinate (D) at (0,2); 
\shade[ball color=red](A) circle (0.025) node[below] {A}; 
\shade[ball color=red](B) circle (0.025) node[below] {B}; 
\shade[ball color=red](C) circle (0.025) node[below] {C}; 
\shade[ball color=red](D) circle (0.025) node[below] {D}; 
\draw[name path=AB] (A) -- (B); \draw[name path=CD] (C) -- (D); 
\path[name intersections={of=AB and CD}] (intersection-1) coordinate (P); 
\shade[ball color=red](P) circle (0.025) node[below] {P}; 
\end{tikzpicture}
\end{lstlisting}
\begin{center}
	\begin{tikzpicture}[scale=3] 
	\draw[help lines] (0,0) grid (2,2); 
	\coordinate (A) at (0,0); 
	\coordinate (B) at (2,0.5); 
	\coordinate (C) at (2,0); 
	\coordinate (D) at (0,2); 
	\shade[ball color=red](A) circle (0.025) node[below] {A}; 
	\shade[ball color=red](B) circle (0.025) node[below] {B}; 
	\shade[ball color=red](C) circle (0.025) node[below] {C}; 
	\shade[ball color=red](D) circle (0.025) node[below] {D}; 
	\draw[name path=AB] (A) -- (B); \draw[name path=CD] (C) -- (D); 
	\path[name intersections={of=AB and CD}] (intersection-1) coordinate (P); 
	\shade[ball color=red](P) circle (0.025) node[below] {P}; 
	\end{tikzpicture}
\end{center}

这个例子用到了点的定义,点的标出,以及path交点的定义,要用到library: intersections 。有时候有些路径你不希望显示出来那么就用path命令来定义路径。

给新交点取名字:用 by 选项可以给画出来的交点取一个名字,默认的 intersection-1 之类的也可以使用。此外还可以加上选项:

\begin{Verbatim} 
\path[name intersections={of=D and E,by={[label=above:$C$]C,[label=below:$C'$]C'}}]; 
\end{Verbatim}
\section{圆(Circle Path)}
我们已经知道如何使用 tikz 在行内画图, 下面我们用以下代码在文中画出图
\begin{lstlisting}
\begin{tikzpicture} 
\draw[step=1,color=gray!40] (-2,-2) grid (2,2); 
\draw[->] (-3,0) -- (3,0); 
\draw[->] (0,-3) -- (0,3); 
\draw (0,0) circle (1);  
\end{tikzpicture}
\end{lstlisting}
\begin{center}
	\begin{tikzpicture} 
	\draw[step=1,color=gray!40] (-2,-2) grid (2,2); 
	\draw[->] (-3,0) -- (3,0); 
	\draw[->] (0,-3) -- (0,3); 
	\draw (0,0) circle (1);  
	\end{tikzpicture}
\end{center}

其中第一个点是圆中心,\lstinline{circle}表示画圆,第二个参数是半径大小.

\begin{lstlisting}
\begin{tikzpicture} 
\draw[step=1,color=gray!40] (-2,-2) grid (2,2); 
\draw[->] (-3,0) -- (3,0); 
\draw[->] (0,-3) -- (0,3); 
\draw (0,0) ellipse (1 and 0.5); 
\end{tikzpicture}
\end{lstlisting}
\begin{center}
	\begin{tikzpicture} 
	\draw[step=1,color=gray!40] (-2,-2) grid (2,2); 
	\draw[->] (-3,0) -- (3,0); 
	\draw[->] (0,-3) -- (0,3); 
	\draw (0,0) ellipse (1 and 0.5); 
	\end{tikzpicture}
\end{center}

这里第一个点是椭圆的中心点,ellipse表示画椭圆,后面参数两个值第一个是a也就是椭圆的半长轴,第二个是b也就是椭圆的半短轴。用法:
\begin{Verbatim} 
\draw[options] (x,y) circle (raidus);
\draw[options] (x,y) ellipse (x.raidus anda y.radius);
\end{Verbatim}

\begin{lstlisting}
\begin{tikzpicture}
\draw (0,0) circle (10pt);
\draw[red] (1,0) circle (15pt);
\draw[fill=red] (2,0) circle (20pt);
\draw[red,fill=red] (3,0) ellipse (20pt and 25pt);
\filldraw[blue,rotate=30] (3.5,-2) ellipse (25pt and 30pt); % another way
\end{tikzpicture}

\begin{tikzpicture}
\draw (- 20pt,0) node [auto] {(a)};
\draw [thick ,blue ] (0pt ,0pt ) circle [radius =10 pt ] ;

\draw (30pt,0) node [auto] {(b)};
\draw [rotate =15] (60pt ,0) ellipse [x radius =20 pt , y radius =10 pt ];
\end{tikzpicture}
\tikz {\draw (0,0) circle (1);\fill [green](0,0) circle (1);}
\tikz \filldraw [red] (0,0) ellipse (2 and 1);
\begin{tikzpicture}[line width=1pt]
\draw (0,0)circle(1.0) (1,0)circle(1.0)
(60:1)circle(1.0);
\clip (0,0) circle (1.0);
\clip (1,0) circle (1.0);
\fill[blue] (60:1)circle(1.0);
\end{tikzpicture}
\begin{center}
\tikz {\draw (0,0)coordinate(D) node[below left=-0.5pt and -4pt]{$D$} --(0,1)coordinate(A) node[above left=-0.5pt and -4pt]{$A$}--(2,1)coordinate(B) node[above]{$B$} -- (1, 1) arc [start angle=90, end angle=180, radius=1] rectangle (2,1)--(0,0)-- (2, 0)coordinate(C) node[below]{$C$} ;
\fill [green](0, 0) -- (2, 1) -- (1, 1)arc [start angle=90, end angle=180, radius=1] -- cycle;
\clip (0,0) rectangle (2,1);
\draw (1,0) circle(1);
\draw (0,0) -- (2,0);}
\end{center}
\end{lstlisting}
\begin{center}
\begin{tikzpicture}
\draw (0,0) circle (10pt);
\draw[red] (1,0) circle (15pt);
\draw[fill=red] (3,0) circle (20pt);
\draw[red,fill=red] (5,0) ellipse (20pt and 25pt);
\filldraw[blue,rotate=30] (6.5,-2) ellipse (25pt and 30pt); % another way
\end{tikzpicture}
\end{center}
\begin{tikzpicture}
\draw (- 20pt,0) node [auto] {(a)};
\draw [thick ,blue ] (0pt ,0pt ) circle [radius =10 pt ] ;
\draw (30pt,0) node [auto] {(b)};
\draw [rotate =15] (60pt ,0) ellipse [x radius =20 pt , y radius =10 pt ];
\end{tikzpicture}
\tikz {\draw (0,0) circle (1);\fill [green](0,0) circle (1);}
\tikz \filldraw [red] (0,0) ellipse (2 and 1);
\begin{tikzpicture}[line width=1pt]
\draw (0,0)circle(1.0) (1,0)circle(1.0)
(60:1)circle(1.0);
\clip (0,0) circle (1.0);
\clip (1,0) circle (1.0);
\fill[blue] (60:1)circle(1.0);
\end{tikzpicture}
\begin{center}
\tikz {\draw (0,0)coordinate(D) node[below left=-0.5pt and -4pt]{$D$} --(0,1)coordinate(A) node[above left=-0.5pt and -4pt]{$A$}--(2,1)coordinate(B) node[above]{$B$} -- (1, 1) arc [start angle=90, end angle=180, radius=1] rectangle (2,1)--(0,0)-- (2, 0)coordinate(C) node[below]{$C$} ;
	\fill [green](0, 0) -- (2, 1) -- (1, 1)arc [start angle=90, end angle=180, radius=1] -- cycle;
	\clip (0,0) rectangle (2,1);
	\draw (1,0) circle(1);
	\draw (0,0) -- (2,0);}
\end{center}
\begin{lstlisting}
\begin{tikzpicture}[scale=0.5,thick]
\def\a{5}%长半轴
\def\b{3}%短半轴
\def\c{4}%焦半轴
\def\ptsize{2.0pt} %点的半径

%x 轴 和 y轴 
\path[name path=xaxis,thick,draw,->](-6,0)--(6,0) node[right] {$x$};
\path[name path=yaxis,thick,draw,->](0,-6)--(0,6) node[above,right] {$y$};
%x 轴 与 y 轴的交点 
\path [name intersections={of = xaxis and yaxis}];
\coordinate[label=below right:$O$] (O) at (intersection-1);
%画一个椭圆
\draw [name path = myellipse ] (intersection-1) ellipse (\a cm and \b cm);
%椭圆与x 轴交点
\path [name intersections={of = xaxis and myellipse}];
\coordinate[label=below right:$A_2$] (a2) at (intersection-1);
\coordinate[label=below left:$A_1$] (a1) at (intersection-2);
%椭圆与y 轴交点
\path [name intersections={of = yaxis and myellipse}];
\coordinate[label=above right:$B_2$] (b2) at (intersection-1);
\coordinate[label=below right:$B_1$] (b1) at (intersection-2);
%焦点
\coordinate[label=below :$F_1$] (f1) at (-\c,0);
\coordinate[label=below :$F_2$] (f2) at (\c,0);
%a b c 的几何意义
\draw (b2) --(f2) node[midway,above] {$a$};
\draw (b2) --(O) node[midway,left] {$b$};
\draw (O) --(f2) node[midway,below] {$c$};
%阴影部分填充
\fill [pattern =north west lines,pattern color = black!70](b2)--(O)--(f2)--cycle;
%画点
\foreach \p in {O,a1,a2,b1,b2,f1,f2}
\fill (\p) circle (\ptsize);
%在图像右侧再画一个焦点在y轴上的椭圆
\begin{scope}[xshift=13cm]
\path[name path=xaxis,thick,draw,->](-6,0)--(6,0) node[right] {$x$};
\path[name path=yaxis,thick,draw,->](0,-6)--(0,6) node[above,right] {$y$};
\path [name intersections={of = xaxis and yaxis}];
\coordinate[label=below right:$O$] (O) at (intersection-1);
%画一个椭圆
\draw [name path = myellipse ] (intersection-1) ellipse (\b cm and \a cm);

\path [name intersections={of = xaxis and myellipse}];
\coordinate[label=below right:$B_2$] (a2) at (intersection-1);
\coordinate[label=below left:$B_1$] (a1) at (intersection-2);

\path [name intersections={of = yaxis and myellipse}];
\coordinate[label=above right:$A_2$] (b2) at (intersection-1);
\coordinate[label=below right:$A_1$] (b1) at (intersection-2);

\coordinate[label=left :$F_1$] (f1) at (0,-\c);
\coordinate[label=left :$F_2$] (f2) at (0,\c);

\draw (a2) --(f2) node[midway,above] {$a$};
\draw (a2) --(O) node[midway,below] {$b$};
\draw (O) --(f2) node[midway,left] {$c$};

\fill [pattern =north west lines,pattern color = black!70](a2)--(O)--(f2)--cycle;

\foreach \p in {O,a1,a2,b1,b2,f1,f2}
\fill (\p) circle (\ptsize);

\end{scope}
\end{tikzpicture}
\end{lstlisting}
\begin{tikzpicture}[scale=0.5,thick]
\def\a{5}%长半轴
\def\b{3}%短半轴
\def\c{4}%焦半轴
\def\ptsize{2.0pt} %点的半径

%x 轴 和 y轴 
\path[name path=xaxis,thick,draw,->](-6,0)--(6,0) node[right] {$x$};
\path[name path=yaxis,thick,draw,->](0,-6)--(0,6) node[above,right] {$y$};
%x 轴 与 y 轴的交点 
\path [name intersections={of = xaxis and yaxis}];
\coordinate[label=below right:$O$] (O) at (intersection-1);
%画一个椭圆
\draw [name path = myellipse ] (intersection-1) ellipse (\a cm and \b cm);
%椭圆与x 轴交点
\path [name intersections={of = xaxis and myellipse}];
\coordinate[label=below right:$A_2$] (a2) at (intersection-1);
\coordinate[label=below left:$A_1$] (a1) at (intersection-2);
%椭圆与y 轴交点
\path [name intersections={of = yaxis and myellipse}];
\coordinate[label=above right:$B_2$] (b2) at (intersection-1);
\coordinate[label=below right:$B_1$] (b1) at (intersection-2);
%焦点
\coordinate[label=below :$F_1$] (f1) at (-\c,0);
\coordinate[label=below :$F_2$] (f2) at (\c,0);
%a b c 的几何意义
\draw (b2) --(f2) node[midway,above] {$a$};
\draw (b2) --(O) node[midway,left] {$b$};
\draw (O) --(f2) node[midway,below] {$c$};
%阴影部分填充
\fill [pattern =north west lines,pattern color = black!70](b2)--(O)--(f2)--cycle;
%画点
\foreach \p in {O,a1,a2,b1,b2,f1,f2}
\fill (\p) circle (\ptsize);
%在图像右侧再画一个焦点在y轴上的椭圆
\begin{scope}[xshift=13cm]
\path[name path=xaxis,thick,draw,->](-6,0)--(6,0) node[right] {$x$};
\path[name path=yaxis,thick,draw,->](0,-6)--(0,6) node[above,right] {$y$};
\path [name intersections={of = xaxis and yaxis}];
\coordinate[label=below right:$O$] (O) at (intersection-1);
%画一个椭圆
\draw [name path = myellipse ] (intersection-1) ellipse (\b cm and \a cm);

\path [name intersections={of = xaxis and myellipse}];
\coordinate[label=below right:$B_2$] (a2) at (intersection-1);
\coordinate[label=below left:$B_1$] (a1) at (intersection-2);

\path [name intersections={of = yaxis and myellipse}];
\coordinate[label=above right:$A_2$] (b2) at (intersection-1);
\coordinate[label=below right:$A_1$] (b1) at (intersection-2);

\coordinate[label=left :$F_1$] (f1) at (0,-\c);
\coordinate[label=left :$F_2$] (f2) at (0,\c);

\draw (a2) --(f2) node[midway,above] {$a$};
\draw (a2) --(O) node[midway,below] {$b$};
\draw (O) --(f2) node[midway,left] {$c$};

\fill [pattern =north west lines,pattern color = black!70](a2)--(O)--(f2)--cycle;

\foreach \p in {O,a1,a2,b1,b2,f1,f2}
\fill (\p) circle (\ptsize);

\end{scope}

\end{tikzpicture}
\begin{lstlisting}
\begin{tikzpicture}
\draw(3,-1) coordinate (A) node[right] {A}
-- (0,0) coordinate (B) node[left] {B}
-- (2,2) coordinate (C) node[above right] {C}
pic["$60\circ$", draw=orange, <->, angle eccentricity=1.2, angle radius=1cm]
{angle=A--B--C};
\end{tikzpicture}

\begin{tikzpicture}
\draw(0,0) circle(4cm);  %画一个4cm的圆圈
%把圆的右半边填黑
\begin{scope}
\clip(0,0) circle(4cm);
\fill[black] (0,-4) rectangle (4,4);
\end{scope}
%填黑八卦图左边的半圆
\begin{scope}
\clip(0,2) circle(2cm);
\fill[black] (-4,0) rectangle (4,4);
\end{scope}
%填白八卦图右边的半圆
\begin{scope}
\clip(0,-2) circle(2cm);
\fill[white] (-4,0) rectangle (4,-4);
\end{scope}
%把黑色部分的小圆圈填为白色
\begin{scope}
\clip(0,2) circle(0.5cm);
\fill[white] (-4,0) rectangle (4,4);
\end{scope}
%绘制下面的白色圆圈
\draw(0,-2) circle(0.5cm);
\end{tikzpicture}
\end{lstlisting}
\begin{tikzpicture}
\draw(3,-1) coordinate (A) node[right] {A}
-- (0,0) coordinate (B) node[left] {B}
-- (2,2) coordinate (C) node[above right] {C}
pic["$60\circ$", draw=orange, <->, angle eccentricity=1.2, angle radius=1cm]
{angle=A--B--C};
\end{tikzpicture}
\begin{tikzpicture}
\draw(0,0) circle(4cm);  %画一个4cm的圆圈
%把圆的右半边填黑
\begin{scope}
\clip(0,0) circle(4cm);
\fill[black] (0,-4) rectangle (4,4);
\end{scope}
%填黑八卦图左边的半圆
\begin{scope}
\clip(0,2) circle(2cm);
\fill[black] (-4,0) rectangle (4,4);
\end{scope}
%填白八卦图右边的半圆
\begin{scope}
\clip(0,-2) circle(2cm);
\fill[white] (-4,0) rectangle (4,-4);
\end{scope}
%把黑色部分的小圆圈填为白色
\begin{scope}
\clip(0,2) circle(0.5cm);
\fill[white] (-4,0) rectangle (4,4);
\end{scope}
%绘制下面的白色圆圈
\draw(0,-2) circle(0.5cm);
\end{tikzpicture}
\begin{lstlisting}
\begin{center}
\begin{tikzpicture}
\foreach \x in {0,1,...,10} {
\node[draw, circle, inner sep=0pt, minimum size=\x mm] {};
}
\foreach \a/\x/\y in {1/orange/八一杯, 2/teal/团子杯, 3/magenta/熊赛} {
\node[circle, minimum size=2*\a mm, draw=\x!90, fill=\x!30, text=\x!70] at (2*\a, 0) {\y};
}
\end{tikzpicture}

\begin{tikzpicture}
\tikzstyle{main}=[circle, minimum size = 10mm, thick, draw =black!80, node distance = 16mm]
\tikzstyle{connect}=[-latex, thick]
\tikzstyle{box}=[rectangle, draw=black!100]
\node[main, fill = white!100] (z) [label=below:\emph{z}] { };
\node[main, fill = black!50] (d) [left=of z,label=below:$d$] { };
\node[main, fill = black!50] (w) [right=of z,label=below:$w$] {};
\path (d) edge [connect] (z)
(z) edge [connect] (w);
\node[rectangle, inner sep=0mm, fit= (z) (w),label=below right:$N$, xshift=13mm] {};
\node[rectangle, inner sep=5.6mm,draw=black!100, fit= (z) (w)] {};
\node[rectangle, inner sep=4.8mm, fit= (z) (w),label=below right:$M$, xshift=12.5mm] {};
\node[rectangle, inner sep=10.4mm, draw=black!100, fit = (d) (z) (w)] {};
\end{tikzpicture}
\end{center}
\end{lstlisting}
\begin{center}
	\begin{tikzpicture}
	\foreach \x in {0,1,...,10} {
		\node[draw, circle, inner sep=0pt, minimum size=\x mm] {};
	}
	\foreach \a/\x/\y in {1/orange/八一杯, 2/teal/团子杯, 3/magenta/熊赛} {
		\node[circle, minimum size=2*\a mm, draw=\x!90, fill=\x!30, text=\x!70] at (2*\a, 0) {\y};
	}
\end{tikzpicture}
	
\begin{tikzpicture}
	\tikzstyle{main}=[circle, minimum size = 10mm, thick, draw =black!80, node distance = 16mm]
	\tikzstyle{connect}=[-latex, thick]
	\tikzstyle{box}=[rectangle, draw=black!100]
	\node[main, fill = white!100] (z) [label=below:\emph{z}] { };
	\node[main, fill = black!50] (d) [left=of z,label=below:$d$] { };
	\node[main, fill = black!50] (w) [right=of z,label=below:$w$] {};
	\path (d) edge [connect] (z)
	(z) edge [connect] (w);
	\node[rectangle, inner sep=0mm, fit= (z) (w),label=below right:$N$, xshift=13mm] {};
	\node[rectangle, inner sep=5.6mm,draw=black!100, fit= (z) (w)] {};
	\node[rectangle, inner sep=4.8mm, fit= (z) (w),label=below right:$M$, xshift=12.5mm] {};
	\node[rectangle, inner sep=10.4mm, draw=black!100, fit = (d) (z) (w)] {};
\end{tikzpicture}
\end{center}
\section{矩形与多边形(Rectangle Path)}
画正方形或矩形使用关键字 rectangle ,左下角和右上角坐标控制整个矩形的形状。如果将矩形旋转,则旋转角度以右下角坐标为轴心,逆时针方向旋转。用法如下:
\begin{verbatim}
 	\draw[options] (x1,y1) rectangle (x2,y2);
	\draw[options] (x,y) rectangle +(width,height);
\end{verbatim}
\begin{lstlisting}
\begin{tikzpicture}
\draw (-1,0) -- (1,0);
\draw (0,-1) -- (0,1);
\draw[rotate=30, fill=red] (-0.5,-0.5) rectangle (0.5,0.5);
\draw (2,-0.5) rectangle +(1,1);
\end{tikzpicture}
\begin{tikzpicture}[scale=1.5]
% 第一个坐标为左下角,第二个坐标为右上角。
\filldraw [thick ,cyan ] (0.3,0) rectangle (0.8, 0.5);
% 旋转角度以右下角坐标为轴心 ,逆时针旋转。
\filldraw [thick ,yellow ,rotate =15] (- 0.5,0) rectangle (0,0.5);
\draw [thick ,green ] (- 0.5,0) rectangle (0,0.5);
\end{tikzpicture}
\begin{tikzpicture}[line width=2pt]
\draw (0,0) --(1,1) --(2,0) --cycle;
\fill[blue] (0,0) --(1,1) --(2,0) --cycle;
\end{tikzpicture}
\tikz \draw (0,0) rectangle (2,1);
\tikz \draw [step=0.5] (0,0) grid (3,2);
\tikz {\draw (0,0) circle (1);\fill [green](0,0) circle (1);}
\end{lstlisting}
\begin{tikzpicture}
\draw (-1,0) -- (1,0);
\draw (0,-1) -- (0,1);
\draw[rotate=30, fill=red] (-0.5,-0.5) rectangle (0.5,0.5);
\draw (2,-0.5) rectangle +(1,1);
\end{tikzpicture}
\begin{tikzpicture}[scale=1.5]
% 第一个坐标为左下角,第二个坐标为右上角。
\filldraw [thick ,cyan ] (0.3,0) rectangle (0.8, 0.5);
% 旋转角度以右下角坐标为轴心 ,逆时针旋转。
\filldraw [thick ,yellow ,rotate =15] (- 0.5,0) rectangle (0,0.5);
\draw [thick ,green ] (- 0.5,0) rectangle (0,0.5);
\end{tikzpicture}
\begin{tikzpicture}[line width=2pt]
\draw (0,0) --(1,1) --(2,0) --cycle;
\fill[blue] (0,0) --(1,1) --(2,0) --cycle;
\end{tikzpicture}
\tikz \draw (0,0) rectangle (2,1);
\tikz \draw [step=0.5] (0,0) grid (3,2);
\tikz {\draw (0,0) circle (1);\fill [green](0,0) circle (1);}
\begin{lstlisting}
\begin{tikzpicture}[scale=2]
\draw[step=1,color=gray!40] (-2,-2) grid (2,2);
\draw[color=red] (-1,-1) rectangle (1,1);
\end{tikzpicture}
\end{lstlisting}
\begin{center}
	\begin{tikzpicture}[scale=2]
	\draw[step=1,color=gray!40] (-2,-2) grid (2,2);
	\draw[color=red] (-1,-1) rectangle (1,1);
	\end{tikzpicture}
\end{center}

这里使用了可选项 color=red 来控制线条的颜色,然后画长方形的第一个点是左底点,rectangle表示画长方形,第二个点表示右顶点。如果想要放大图形,可在tikzpicture环境后面跟上可选项 [scale=2] ,即将图形放大两倍。
\begin{lstlisting}
\begin{tikzpicture}
\draw (0,0) circle (4) ;
\coordinate (O) at (0,0);
\shade[ball color=red](O) circle (0.1) node[below] {O};
\def\n{5}
\pgfmathsetmacro\i{\n-1}
\foreach \x in {0,...,\i}
{
	\def\pointname{\x}
	\coordinate (\pointname) at ($(0,0) +(\x*360/\n:4cm)$)  ;
	\shade[ball color=red](\pointname) circle (0.05) node[below] {\small \x};
}

\draw (0)
\foreach \x in {0,...,\i}
{ -- (\x) } -- cycle;

\end{tikzpicture}
\end{lstlisting}
\begin{center}
\begin{tikzpicture}
\draw (0,0) circle (4) ;
\coordinate (O) at (0,0);
\shade[ball color=red](O) circle (0.1) node[below] {O};
\def\n{5}
\pgfmathsetmacro\i{\n-1}
\foreach \x in {0,...,\i}
{
\def\pointname{\x}
\coordinate (\pointname) at ($(0,0) +(\x*360/\n:4cm)$)  ;
\shade[ball color=red](\pointname) circle (0.05) node[below] {\small \x};
}

\draw (0)
\foreach \x in {0,...,\i}
{ -- (\x) } -- cycle;
\end{tikzpicture}
\end{center}
这个例子核心内容是批量定义点和点的运算,把这个弄懂了,后面tikz的核心大门就为你打开了,然后很多图形都可以用简洁的命令生成出来了。
\section{网格(Grid Path)}
网格主要用于辅助绘图,其中 help lines 是个不错的参数设置。其用法:
\begin{verbatim}
\draw[options] (x1,y1) grid (x2,y2); 
\end{verbatim}
\begin{lstlisting}
\begin{tikzpicture}
\draw[step=1.0cm, gray, very thin] (-1.4,-1.4) grid (1.4,1.4);
\draw (-1.5,0) -- (1.5,0);
\draw (0,-1.5) -- (0,1.5);
\draw (0,0) circle (1cm);
\draw[step=2pt] (0,0) grid (30pt,30pt);
\end{tikzpicture}
\end{lstlisting}
\begin{center}
 \begin{tikzpicture}
\draw[step=1.0cm, gray, very thin] (-1.4,-1.4) grid (1.4,1.4);
\draw (-1.5,0) -- (1.5,0);
\draw (0,-1.5) -- (0,1.5);
\draw (0,0) circle (1cm);
\draw[step=2pt] (0,0) grid (30pt,30pt);
\end{tikzpicture}
\end{center}
此外 step 用来控制网格之间的间距,可以color来设置网格的颜色,不过一般没那个必要。然后接下来第一个坐标点是网格的左底点,第二个坐标点是网格的右定点。
\begin{lstlisting}
\begin{tikzpicture} 
\draw[help lines] ( -5,-5 ) grid ( 5, 5); 
\end{tikzpicture}
\end{lstlisting}
\begin{center}
	\begin{tikzpicture} 
	\draw[help lines] ( -5,-5 ) grid ( 5, 5); 
	\end{tikzpicture}
\end{center}

我们看到tikz的每一条命令最后都要跟一个分号";"。
\section{直线(Straight Path)}
直线就是两个坐标点相连,中间 -- 符号表示直线的意思。用法如下:
\begin{verbatim}
\draw[options] (x1,y1) -- (x2,y2) -- (x3,y3);
\end{verbatim}
\begin{lstlisting}
\begin{tikzpicture}
\draw (-1.5,0) -- (1.5,0) -- (0,-1.5) -- (0,1.5);
\draw[thick, rounded corners=10pt] (0,0) -- (0,2) -- (1,3.25) -- 
(2,2) -- (2,0) -- (0,2) -- (2,2) -- (0,0) -- (2,0);
\end{tikzpicture}
\end{lstlisting}
\begin{center}
\begin{tikzpicture}
\draw (-1.5,0) -- (1.5,0) -- (0,-1.5) -- (0,1.5);
\draw[thick, rounded corners=10pt] (0,0) -- (0,2) -- (1,3.25) -- 
(2,2) -- (2,0) -- (0,2) -- (2,2) -- (0,0) -- (2,0);
\end{tikzpicture}
\end{center}
之前网格是grid表示网格的意思。如果几个点用 -- 符号连接起来,表示这几个点连着来画几条折线,有多个画直线命令依次执行的意思。
\begin{lstlisting}
\begin{tikzpicture} 
\draw[help lines] ( -4,-4 ) grid (4,4); 
\draw[red] (-3,0) -- (3,0); 
\draw[red] (0,-3) -- (0,3); 
\end{tikzpicture}
\end{lstlisting}
\begin{center}
\begin{tikzpicture} 
\draw[help lines] ( -4,-4 ) grid (4,4); 
\draw[red] (-3,0) -- (3,0); 
\draw[red] (0,-3) -- (0,3); 
\end{tikzpicture}
\end{center}
如果在直线带上箭头,draw命令可以跟上可选项 \lstinline{->} ,这样直线的右端就有一个箭头了。此外还有: \lstinline{->>},\lstinline {->|} , \lstinline{-to},\lstinline {-latex} ,\lstinline {-stealth} ,他们的效果从上到下依次演示如下:
\begin{lstlisting}
\begin{tikzpicture} 
\draw[->] (-3,3) -- (3,3); 
\draw[->>] (-3,2) -- (3,2); 
\draw[->|] (-3,1) -- (3,1); 
\draw[-to] (-3,0) -- (3,0); 
\draw[-latex] (-3,-1) -- (3,-1); 
\draw[-stealth] (-3,-2) -- (3,-2); 
\end{tikzpicture}
\end{lstlisting}
\begin{center}
\begin{tikzpicture} 
\draw[->] (-3,3) -- (3,3); 
\draw[->>] (-3,2) -- (3,2); 
\draw[->|] (-3,1) -- (3,1); 
\draw[-to] (-3,0) -- (3,0); 
\draw[-latex] (-3,-1) -- (3,-1); 
\draw[-stealth] (-3,-2) -- (3,-2); 
\end{tikzpicture}
\end{center}
\section{抛物线(Parabola Path)}
画实例用的抛物线使用关键字 parabola,用法如下:
\begin{verbatim}
\draw[options] (x1,y1) parabola (x2,y2);
\end{verbatim}

\tikz \draw (0,0) rectangle (1,1) (0,0) parabola (1,1);再画两个
\tikz \draw[x=0.2cm,y=0.2cm] (0,0) parabola bend (4,10) (6,6);\tikz \draw (0,0) sin (1,1) cos (2,0) sin (3,-1) cos (4,0);

第一个点 (0,0) 是起点,使用 sin画到第二点 (1,1) 。如果没有给出起点,那么 cos 会以 (1,1) 为起点,画到点 (2,0)。[x=1ex,y=1ex] 是表示在一个行距高度内,宽度为 1.57 倍的行距内画图。
section{箭头}
Tikz 默认的箭头有点难看,主要是因为它指的不精确。一个名字叫作“悄悄地 (stealth) ”的箭头。
\tikz{
	\draw [line width=1mm, -{Stealth[length=10mm, open]}]
	(0,0) -- (2,0);
	\draw [|<->|] (2,.6) -- node[above=1mm] {10mm} ++(-10mm,0);
}
\begin{tikzpicture}[>= stealth]
\draw [->, red] (0,0) arc [start angle =180, end angle=30,radius=1cm];
\draw [<<- ,very thick] (1,0) -- (1.5cm, 10pt) -- (2cm,0pt) -- (2.5cm,10pt);
\end{tikzpicture}

以及Arc Path的使用来绘制曲线,用法:
\begin{verbatim}
\draw (x,y) arc (angle1:angle2:radius);
\draw (x,y) arc [start angle=angle1, end angle=angle2, radius=radius];
\draw (x,y) arc (angle1:angle2:x.radius and y.radius);
\draw (x,y) arc [start angle=angle1, end angle=angle2, x radius=rx, y radius=ry]
\end{verbatim}
\begin{lstlisting}
\begin{tikzpicture}
\draw (-1.5,0) -- (1.5,0);
\draw (0,-1.5) -- (0,1.5);
\draw (0.5,0) arc (0:120:0.5cm);
\draw (1,0) arc (0:315:1.75cm and 1cm);
\draw (-1,0) arc [start angle=180, end angle=120, radius=0.5cm];
% 以上不是推荐的方式
\end{tikzpicture}
\end{lstlisting}
\begin{center}
	\begin{tikzpicture}
	\draw (-1.5,0) -- (1.5,0);
	\draw (0,-1.5) -- (0,1.5);
	\draw (0.5,0) arc (0:120:0.5cm);
	\draw (1,0) arc (0:315:1.75cm and 1cm);
	\draw (-1,0) arc [start angle=180, end angle=120, radius=0.5cm];
	% 以上不是推荐的方式
	\end{tikzpicture}
\end{center}

正弦或余弦曲线使用关键字sin/cos,用法:
\begin{verbatim}
\draw[options] (x1,y1) sin (x2,y2);
\draw[options] (x1,y1) cos (x2,y2);
\end{verbatim} 
\begin{lstlisting}
 \begin{tikzpicture}
\draw[help lines] (-0.5,-1.5) grid (4.5,1.5);
\draw[red] (0,0) sin (1,1) cos (2,0) sin (3,-1) cos (4,0);
\draw[blue] (0,1) cos (1,0) sin (2,-1) cos (3,0) sin (4,1);
\end{tikzpicture}
\end{lstlisting}
\begin{center}
	 \begin{tikzpicture}
	\draw[help lines] (-0.5,-1.5) grid (4.5,1.5);
	\draw[red] (0,0) sin (1,1) cos (2,0) sin (3,-1) cos (4,0);
	\draw[blue] (0,1) cos (1,0) sin (2,-1) cos (3,0) sin (4,1);
	\end{tikzpicture}
\end{center}
\begin{lstlisting}
\begin{tikzpicture}[scale=1.6]%scale参数可以使得图形放大一定的倍数而本身的字体大小可以保持不变。
\def\iangle{120}
%画左边的圆
%scope环境里够成一整个区块,然后可以使这一整个区块进行平移。
\begin{scope}[xshift=-2cm]
\draw[->] (-1.2,0) --(1.2,0);
\draw[->] (0,-1.2) --(0,1.2);
\draw[thick] (0,0) circle(1cm);
\coordinate [label = \iangle:$P$] (P) at (\iangle:1);
\coordinate [label = below:$P0$] (P0) at (P |-0,0);
\draw (P)--(P0);
\draw (0,0)--(P);
\fill [fill = gray,fill opacity=0.2] (0,0)--(0:1) arc (0:\iangle:1)--cycle;
\filldraw [fill = gray,fill opacity=0.5] (0,0)--(0:0.3) arc (0:\iangle:0.3)--cycle;
\node [right] at (\iangle/2:0.3) {\ang{\iangle}};

\end{scope}
%画右边的正弦曲线
\draw[->] (0,0) --({rad(210)},0);
\draw[->] (0,-1.2) --(0,1.2);
\draw [thick,domain=0:rad(210)] plot (\x,{sin(\x r)}) node [right] {$\sin x$};

\foreach \t in {0,90,180}
{
\draw ({rad(\t)},-0.05)--({rad(\t)},0.05) ;
\node [below,outer sep =2pt ,font=\small , fill = white] at ({rad(\t)},0) {\ang{\t}};
}
\foreach \y in {-1,1}
{
\draw (-0.05,\y) -- (0.05,\y);
}
\coordinate [label=above:{$Q$}] (Q) at ({rad(\iangle)},{sin(\iangle)});
\coordinate [label=below:{$Q_0$}] (Q0) at (Q |- 0,0);
\draw (Q)--(Q0);
%左右相互连接
\draw[dashed] (P) --(Q);
\end{tikzpicture}
\end{lstlisting}
\begin{center}
	\begin{tikzpicture}[scale=1.6]%scale参数可以使得图形放大一定的倍数而本身的字体大小可以保持不变。
	\def\iangle{120}
	%画左边的圆
	%scope环境里够成一整个区块,然后可以使这一整个区块进行平移。
	\begin{scope}[xshift=-2cm]
	\draw[->] (-1.2,0) --(1.2,0);
	\draw[->] (0,-1.2) --(0,1.2);
	\draw[thick] (0,0) circle(1cm);
	\coordinate [label = \iangle:$P$] (P) at (\iangle:1);
	\coordinate [label = below:$P0$] (P0) at (P |-0,0);
	\draw (P)--(P0);
	\draw (0,0)--(P);
	\fill [fill = gray,fill opacity=0.2] (0,0)--(0:1) arc (0:\iangle:1)--cycle;
	\filldraw [fill = gray,fill opacity=0.5] (0,0)--(0:0.3) arc (0:\iangle:0.3)--cycle;
	\node [right] at (\iangle/2:0.3) {\ang{\iangle}};
	
	\end{scope}
	%画右边的正弦曲线
	\draw[->] (0,0) --({rad(210)},0);
	\draw[->] (0,-1.2) --(0,1.2);
	\draw [thick,domain=0:rad(210)] plot (\x,{sin(\x r)}) node [right] {$\sin x$};
	
	\foreach \t in {0,90,180}
	{
		\draw ({rad(\t)},-0.05)--({rad(\t)},0.05) ;
		\node [below,outer sep =2pt ,font=\small , fill = white] at ({rad(\t)},0) {\ang{\t}};
	}
	\foreach \y in {-1,1}
	{
		\draw (-0.05,\y) -- (0.05,\y);
	}
	\coordinate [label=above:{$Q$}] (Q) at ({rad(\iangle)},{sin(\iangle)});
	\coordinate [label=below:{$Q_0$}] (Q0) at (Q |- 0,0);
	\draw (Q)--(Q0);
	%左右相互连接
	\draw[dashed] (P) --(Q);
	\end{tikzpicture}
\end{center}
\section{添加文本(Add text)}
添加文字比较简单,在已知位置上用 node 标出文字即可。其用法:
\begin{verbatim}
\draw (x,y) node[options] {text};
\draw (x,y) node[options] {text};
\node[options] at (x,y) {text};
options: above,below,left,right, or anchor=north,south,west,east.
\end{verbatim}
\begin{lstlisting}
\begin{tikzpicture}[scale=2]
\draw[<->] (0,1) -- (0,0) -- (1,0);
\draw[fill] (0.5,0.5) circle (0.05);
\draw (0.5,0.5) node[above=10pt] {A} node[left=10pt] {L} 
node[below=10pt] {B} node[right=10pt] {R};
\draw (0.5,0.5) node[above left=2pt] {AL} node[below left=2pt] {BL} 
node[below right=2pt] {BR} node[above right=2pt] {AR};
\end{tikzpicture}
\begin{tikzpicture}[scale=2]
\draw[<->] (0,1) -- (0,0) -- (1,0);
\draw[fill] (0.5,0.5) circle (0.05);
\draw (0.5,0.5) node[anchor=north] {N} node[anchor=west] {W} 
node[anchor=south] {S} node[anchor=east] {E};
\draw (0.5,0.5) node[anchor=north west] {NW} node[anchor=south west] {SW} 
node[anchor=south east] {SE} node[anchor=north east] {NE};
\end{tikzpicture}
\end{lstlisting}
\begin{center}
	\begin{tikzpicture}[scale=2]
	\draw[<->] (0,1) -- (0,0) -- (1,0);
	\draw[fill] (0.5,0.5) circle (0.05);
	\draw (0.5,0.5) node[above=10pt] {A} node[left=10pt] {L} 
	node[below=10pt] {B} node[right=10pt] {R};
	\draw (0.5,0.5) node[above left=2pt] {AL} node[below left=2pt] {BL} 
	node[below right=2pt] {BR} node[above right=2pt] {AR};
	\end{tikzpicture}
	\begin{tikzpicture}[scale=2]
	\draw[<->] (0,1) -- (0,0) -- (1,0);
	\draw[fill] (0.5,0.5) circle (0.05);
	\draw (0.5,0.5) node[anchor=north] {N} node[anchor=west] {W} 
	node[anchor=south] {S} node[anchor=east] {E};
	\draw (0.5,0.5) node[anchor=north west] {NW} node[anchor=south west] {SW} 
	node[anchor=south east] {SE} node[anchor=north east] {NE};
	\end{tikzpicture}
\end{center}
\begin{lstlisting}
\begin{tikzpicture}
\draw (0,0) .. controls (6,1) and (9,1) .. 
node[near start,sloped,above] {near start} 
node {midway} 
node[very near end,sloped,below] {very near end} (12,0);
\end{tikzpicture}
\end{lstlisting}
\begin{center}
	\begin{tikzpicture}
	\draw (0,0) .. controls (6,1) and (9,1) .. 
	node[near start,sloped,above] {near start} 
	node {midway} 
	node[very near end,sloped,below] {very near end} (12,0);
	\end{tikzpicture}
\end{center}
\begin{verbatim}
\begin{tikzpicture}
\draw (0,0) node(a) [draw,align=center] {This is a test\\$a$ node} 
(1,1) node(b) [draw] {Node B};
\draw (a.north) |- (b.west);
\draw[color=red] (a.east) -| (2,1.5) -| (b.north);
\end{tikzpicture}
\end{verbatim}
\begin{center}
	\begin{tikzpicture}
	\draw (0,0) node(a) [draw,align=center] {This is a test\\$a$ node} 
	(1,1) node(b) [draw] {Node B};
	\draw (a.north) |- (b.west);
	\draw[color=red] (a.east) -| (2,1.5) -| (b.north);
	\end{tikzpicture}
\end{center}
\begin{lstlisting}
\begin{tikzpicture}[scale=2, >= stealth]
\draw[step=0.5cm,gray,very thin] (-1.4,-1.4) grid (1.4,1.4);
\filldraw[fill=green!20!white,draw=red!50!black] (0,0) -- (3mm,0mm)
arc [start angle=0, end angle=30, radius=3mm] -- cycle ;
\draw (2mm, 0.4mm) node {$\alpha$};
\draw[->] (-1.5,0) -- (1.5,0) coordinate (x axis);
\draw[->] (0,-1.5) -- (0,1.5) coordinate (y axis);
\draw (0,0) circle [radius=1cm];
\draw [red,very thick] (30:1cm) -- node[left=1pt,fill=white] {$\sin \alpha$} +(0,-0.5);
\draw [blue,very thick] (30:1cm) ++(0,-0.5) -- node[below=2pt,fill=white] {$\cos \alpha$} (0,0);
\draw [orange,very thick] (1,0) -- (1,{tan(30)}) node[right=1pt,fill=white]
{$\displaystyle \tan \alpha \color{black}=
\frac{{\color{red}\sin \alpha }}{\color{blue}\cos \alpha}$};
\draw (0,0) -- (1,{tan(30)});
\foreach \x/\xtext in {-1,-0.5/-\frac{1}{2}, 1}
\draw [very thick] (\x cm, -1pt) -- (\x cm, 1pt) node[anchor=north,fill=white] {$\xtext$};
\foreach \y in {-1,-0.5,0.5,1}
\draw [very thick] (-1pt,\y cm) -- (1pt,\y cm) node[anchor=east,fill=white] {$\y$};
\end{tikzpicture}
\end{lstlisting}
\begin{center}
	\begin{tikzpicture}[scale=2, >= stealth]
	\draw[step=0.5cm,gray,very thin] (-1.4,-1.4) grid (1.4,1.4);
	\filldraw[fill=green!20!white,draw=red!50!black] (0,0) -- (3mm,0mm)
	arc [start angle=0, end angle=30, radius=3mm] -- cycle ;
	\draw (2mm, 0.4mm) node {$\alpha$};
	\draw[->] (-1.5,0) -- (1.5,0) coordinate (x axis);
	\draw[->] (0,-1.5) -- (0,1.5) coordinate (y axis);
	\draw (0,0) circle [radius=1cm];
	\draw [red,very thick] (30:1cm) -- node[left=1pt,fill=white] {$\sin \alpha$} +(0,-0.5);
	\draw [blue,very thick] (30:1cm) ++(0,-0.5) -- node[below=2pt,fill=white] {$\cos \alpha$} (0,0);
	\draw [orange,very thick] (1,0) -- (1,{tan(30)}) node[right=1pt,fill=white]
	{$\displaystyle \tan \alpha \color{black}=
		\frac{{\color{red}\sin \alpha }}{\color{blue}\cos \alpha}$};
	\draw (0,0) -- (1,{tan(30)});
	\foreach \x/\xtext in {-1,-0.5/-\frac{1}{2}, 1}
	\draw [very thick] (\x cm, -1pt) -- (\x cm, 1pt) node[anchor=north,fill=white] {$\xtext$};
	\foreach \y in {-1,-0.5,0.5,1}
	\draw [very thick] (-1pt,\y cm) -- (1pt,\y cm) node[anchor=east,fill=white] {$\y$};
	\end{tikzpicture}
\end{center}

\section{添加样式(Add Styles)}
\textsf{Styles}是可用于组织图形绘制方式的预定义选项集。要全局定义样式,可以在文档开头使用\textbackslash tikzset命令,用法:
\begin{verbatim}
\tikzset{style_name./style={options}}
\end{verbatim}
\begin{lstlisting}
\tikzset{blue_thin_lines/.style={color=blue!50,very thin}}
\begin{tikzpicture}
\draw[step=0.5cm, blue_thin_lines] (0,0) grid (2,2);
\end{tikzpicture}
\end{lstlisting}
\tikzset{blue_thin_lines/.style={color=blue!50,very thin}}
\begin{center}
\begin{tikzpicture}
\draw[step=0.5cm, blue_thin_lines] (0,0) grid (2,2);
\end{tikzpicture}
\end{center}
为了在局部定义一个样式,我们使用一对方括号 ``[ ]''来定义图片开头的样式,用法:
\begin{verbatim}
[style_name/.style={options}]
\end{verbatim}
\begin{lstlisting}
\begin{tikzpicture}
[red_thick_lines/.style={color=red!50,very thick}];
\draw[step=0.5cm, red_thick_lines] (0,0) grid (2,2);
\end{tikzpicture}
\end{lstlisting}
\begin{center}
	\begin{tikzpicture}
	[red_thick_lines/.style={color=red!50,very thick}];
	\draw[step=0.5cm, red_thick_lines] (0,0) grid (2,2);
	\end{tikzpicture}
\end{center}
也可以分层定义样式,用法:
\begin{verbatim}
\tikzset{style_name1/.style={style_name2, options}}
\end{verbatim}
\begin{lstlisting}
\tikzset{green_help_lines/.style={help lines, color=green!90}}
\begin{tikzpicture}
\draw[step=0.5cm, green_help_lines] (0,0) grid (5,5);
\end{tikzpicture}
\end{lstlisting}
\begin{center}
	\tikzset{green_help_lines/.style={help lines, color=green!90}}
	\begin{tikzpicture}
	\draw[step=0.5cm, green_help_lines] (0,0) grid (5,5);
	\end{tikzpicture}
\end{center}
样式也可以与参数一起使用,用法:
\begin{verbatim}
 [style_name/.style={options},style_name/.default={options}]
\end{verbatim}
\begin{lstlisting}
\begin{tikzpicture}
[para_color/.style={help lines,color=#1!50}, para_color/.default=blue]
\draw[step=0.5cm, para_color] (0,0) grid (2,2);
\draw[step=0.5cm, para_color=red] (2,0) grid (4,2);
\end{tikzpicture}
\end{lstlisting}
\begin{center}
	\begin{tikzpicture}
	[para_color/.style={help lines,color=#1!50}, para_color/.default=blue]
	\draw[step=0.5cm, para_color] (0,0) grid (2,2);
	\draw[step=0.5cm, para_color=red] (2,0) grid (4,2);
	\end{tikzpicture}
\end{center}

预定义一个属性集合,到时候直接赋给相应的实体,TikZ本身就是个宏,因此它为我们提供了强大的属性定义功能,来看这段代码:
\begin{lstlisting}
\begin{tikzpicture}
[LNode/.style={circle,   draw=blue!50, fill=blue!20, very thick, minimum size=10mm}
\node[LNode] (n1) at (0, 0){$\int x \mathrm{d}x$};
\end{tikzpicture}
\end{lstlisting}
\begin{center}
	\begin{tikzpicture}
	[L1Node/.style={circle,  draw=blue!50, fill=blue!20, very thick, minimum size=50mm}
%	L2Node/.style={rectangle,draw=green!50,fill=green!20,very thick, minimum size=50mm}]
	\node[L1Node] (n1) at (0, 0){$\int_{a}^{b} f(x)\mathrm{d}x$};
%	\node[L2Node] (n2) at (2, 0){$n!$};
	\end{tikzpicture}
\end{center}
\section{Draw options and context}
\subsection{绘图选项}
有一些绘图选项可以用来控制颜色、厚度和线条类型。
\begin{itemize}
	\item 颜色: 
	blue \tikz \filldraw[blue] (0,0) rectangle (1.5em,1.5ex);, 
	black \tikz \filldraw[black] (0,0) rectangle (1.5em,1.5ex);, 
	brown \tikz \filldraw[brown] (0,0) rectangle (1.5em,1.5ex);, 
	cyan \tikz \filldraw[cyan] (0,0) rectangle (1.5em,1.5ex);, 
	gray \tikz \filldraw[gray] (0,0) rectangle (1.5em,1.5ex);, 
	green \tikz \filldraw[green] (0,0) rectangle (1.5em,1.5ex);, 
	lightgray \tikz \filldraw[lightgray] (0,0) rectangle (1.5em,1.5ex);, 
	lime \tikz \filldraw[lime] (0,0) rectangle (1.5em,1.5ex);, 
	magenta \tikz \filldraw[magenta] (0,0) rectangle (1.5em,1.5ex);, 
	orange \tikz \filldraw[orange] (0,0) rectangle (1.5em,1.5ex);, 
	pink \tikz \filldraw[pink] (0,0) rectangle (1.5em,1.5ex);, 
	purple \tikz \filldraw[purple] (0,0) rectangle (1.5em,1.5ex);, 
	red \tikz \filldraw[red] (0,0) rectangle (1.5em,1.5ex);, 
	yellow \tikz \filldraw[yellow] (0,0) rectangle (1.5em,1.5ex);, 
	teal \tikz \filldraw[teal] (0,0) rectangle (1.5em,1.5ex);, 
	violet \tikz \filldraw[violet] (0,0) rectangle (1.5em,1.5ex);, 
	white \tikz \draw[fill=white] (0,0) rectangle (1.5em,1.5ex);.\\
	注意:颜色是可以混合, 比如这种颜色命令 \verb|[blue!40!white]| 意味着$40\%$蓝色和$60\%$白色混合在一起。
	\item 厚度: 
	ultra thin \begin{tikzpicture} \filldraw[white] (0,0) rectangle (1em,1.5ex); \draw[ultra thin] (0,0.7ex) -- (1em,0.7ex);\end{tikzpicture}, 
	very thin \begin{tikzpicture} \filldraw[white] (0,0) rectangle (1em,1.5ex); \draw[very thin] (0,0.7ex) -- (1em,0.7ex);\end{tikzpicture}, 
	thin \begin{tikzpicture} \filldraw[white] (0,0) rectangle (1em,1.5ex); \draw[thin] (0,0.7ex) -- (1em,0.7ex);\end{tikzpicture}, 
	semithick \begin{tikzpicture} \filldraw[white] (0,0) rectangle (1em,1.5ex); \draw[semithick] (0,0.7ex) -- (1em,0.7ex);\end{tikzpicture}, 
	thick \begin{tikzpicture} \filldraw[white] (0,0) rectangle (1em,1.5ex); \draw[thick] (0,0.7ex) -- (1em,0.7ex);\end{tikzpicture}, 
	very thick \begin{tikzpicture} \filldraw[white] (0,0) rectangle (1em,1.5ex); \draw[very thick] (0,0.7ex) -- (1em,0.7ex);\end{tikzpicture}, 
	ultra thick \begin{tikzpicture} \filldraw[white] (0,0) rectangle (1em,1.5ex); \draw[ultra thick] (0,0.7ex) -- (1em,0.7ex);\end{tikzpicture}.\\
	注意: \verb![help lines]=[gray,very thin]!. 线厚度也可以通过\verb![line width]! 选择,例如 \verb![line width=0.5cm]!.
	\item 线型: 
	loosely dashed \begin{tikzpicture} \filldraw[white] (0,0) rectangle (1em,1.5ex); \draw[loosely dashed] (0,0.7ex) -- (1em,0.7ex);\end{tikzpicture},
	dashed \begin{tikzpicture} \filldraw[white] (0,0) rectangle (1em,1.5ex); \draw[dashed] (0,0.7ex) -- (1em,0.7ex);\end{tikzpicture},
	densely dashed \begin{tikzpicture} \filldraw[white] (0,0) rectangle (1em,1.5ex); \draw[densely dashed] (0,0.7ex) -- (1em,0.7ex);\end{tikzpicture},
	loosely dotted \begin{tikzpicture} \filldraw[white] (0,0) rectangle (1em,1.5ex); \draw[loosely dotted] (0,0.7ex) -- (1em,0.7ex);\end{tikzpicture},
	dotted \begin{tikzpicture} \filldraw[white] (0,0) rectangle (1em,1.5ex); \draw[dotted] (0,0.7ex) -- (1em,0.7ex);\end{tikzpicture},
	densely dotted \begin{tikzpicture} \filldraw[white] (0,0) rectangle (1em,1.5ex); \draw[densely dotted] (0,0.7ex) -- (1em,0.7ex);\end{tikzpicture}.
	\item 箭头: 
	\verb!<-! \begin{tikzpicture} \filldraw[white] (0,0) rectangle (1em,1.5ex); \draw[<-] (0,0.7ex) -- (1em,0.7ex);\end{tikzpicture},
	\verb!<<-! \begin{tikzpicture} \filldraw[white] (0,0) rectangle (1em,1.5ex); \draw[<<-] (0,0.7ex) -- (1em,0.7ex);\end{tikzpicture},
	\verb!<-|! \begin{tikzpicture} \filldraw[white] (0,0) rectangle (1em,1.5ex); \draw[<-|] (0,0.7ex) -- (1em,0.7ex);\end{tikzpicture},
	\verb!<<-|! \begin{tikzpicture} \filldraw[white] (0,0) rectangle (1em,1.5ex); \draw[<<-|] (0,0.7ex) -- (1em,0.7ex);\end{tikzpicture},
	\verb!->! \begin{tikzpicture} \filldraw[white] (0,0) rectangle (1em,1.5ex); \draw[->] (0,0.7ex) -- (1em,0.7ex);\end{tikzpicture},
	\verb!->>! \begin{tikzpicture} \filldraw[white] (0,0) rectangle (1em,1.5ex); \draw[->>] (0,0.7ex) -- (1em,0.7ex);\end{tikzpicture},
	\verb!|->! \begin{tikzpicture} \filldraw[white] (0,0) rectangle (1em,1.5ex); \draw[|->] (0,0.7ex) -- (1em,0.7ex);\end{tikzpicture},
	\verb!|->>! \begin{tikzpicture} \filldraw[white] (0,0) rectangle (1em,1.5ex); \draw[|->>] (0,0.7ex) -- (1em,0.7ex);\end{tikzpicture},
	\verb!<->! \begin{tikzpicture} \filldraw[white] (0,0) rectangle (1em,1.5ex); \draw[<->] (0,0.7ex) -- (1em,0.7ex);\end{tikzpicture},
	\verb!<<->>! \begin{tikzpicture} \filldraw[white] (0,0) rectangle (1em,1.5ex); \draw[<<->>] (0,0.7ex) -- (1em,0.7ex);\end{tikzpicture}.\\
	注意:你也可以添加\verb!>=stealth!到选项中,它可将箭头更改为'stealth-like'的样式。
\end{itemize}
用法如下:
\begin{verbatim}
 \draw[color, thickness, line type, arrow] (x1,y1) -- (x2,y2);
\end{verbatim}
\begin{lstlisting}
\begin{tikzpicture}
\draw[red, very thin, densely dashed, <-] (0,0) -- (0.9,0);
\draw[green, ultra thick, loosely dotted, |->] (1.1,0) -- (1.9,0);
\draw[blue, semithick, <->, >=stealth] (2.1,0) -- (2.9,0);
\draw[purple, line width=0.3cm] (3.1,0) -- (3.9,0);
\end{tikzpicture}
\end{lstlisting}
\begin{center}
\begin{tikzpicture}
\draw[red, very thin, densely dashed, <-] (0,0) -- (0.9,0);
\draw[green, ultra thick, loosely dotted, |->] (1.1,0) -- (1.9,0);
\draw[blue, semithick, <->, >=stealth] (2.1,0) -- (2.9,0);
\draw[purple, line width=0.3cm] (3.1,0) -- (3.9,0);
\end{tikzpicture}
\end{center}
\subsection{node树}
node结点不但可以用于添加标识,还可以来绘制树形图,下面看一个例子,两个可作个对比,后面对前加了个样式:
\begin{lstlisting}
\begin{tikzpicture}
	\node {root}
	child {node {a1}}
	child {node {a2}
	child {node {b1}}
	child {node {b2}}}
	child {node {a3}};
\end{tikzpicture}
\begin{tikzpicture}
[every node/.style={fill=blue!30,draw=blue!70,rounded corners},
edge from parent/.style={blue,thick,draw}]
	\node {root}
	child {node {a1}}
	child {node {a2}
	child {node {b1}}
	child {node {b2}}}
	child {node {a3}};
\end{tikzpicture}
\end{lstlisting}
\begin{center}
	\begin{tikzpicture}
	\node {root}
	child {node {a1}}
	child {node {a2}
		child {node {b1}}
		child {node {b2}}}
	child {node {a3}};
	\end{tikzpicture}
	\begin{tikzpicture}
	[every node/.style={fill=blue!30,draw=blue!70,rounded corners},
	edge from parent/.style={blue,thick,draw}]
	\node {root}
	child {node {a1}}
	child {node {a2}
		child {node {b1}}
		child {node {b2}}}
	child {node {a3}};
	\end{tikzpicture}
\end{center}
\begin{lstlisting}
\tikzset{place/.style={circle,draw=blue!50,fill=blue!20,
thick,inner sep=0pt,minimum size=6mm}}
\tikzset{transition/.style={rectangle,draw=black!50,
fill=black!20,thick,inner sep=0pt,minimum size=4mm}}
\tikzset{every label/.style=red}
\begin{tikzpicture}[bend angle=45]
\node[place] (waiting)  {};
\node[place] (critical) [below=of waiting] {};
\node[place](semaphore) [below=of critical,label=above:$s\le3$] {};
\node[transition](leave critical) [right=of critical]{};
\node[transition] (enter critical)[left=of critical]{};
\draw [->] (enter critical) to (critical);
\draw [->] (waiting) to [bend right] (enter critical);
\draw [->] (enter critical) to [bend right] (semaphore);
\draw [->] (semaphore) to [bend right] (leave critical);
\draw [->] (critical) to (leave critical);
\draw [->] (leave critical) to [bend right] (waiting);
\end{tikzpicture}
\end{lstlisting}
\tikzset{place/.style={circle,draw=blue!50,fill=blue!20,
		thick,inner sep=0pt,minimum size=6mm}}
\tikzset{transition/.style={rectangle,draw=black!50,
		fill=black!20,thick,inner sep=0pt,minimum size=4mm}}
\tikzset{every label/.style=red}
\begin{center}
\begin{tikzpicture}[bend angle=45]
\node[place] (waiting)  {};
\node[place] (critical) [below=of waiting] {};
\node[place](semaphore) [below=of critical,label=above:$s\le3$] {};
\node[transition](leave critical) [right=of critical]{};
\node[transition] (enter critical)[left=of critical]{};
\draw [->] (enter critical) to (critical);
\draw [->] (waiting) to [bend right] (enter critical);
\draw [->] (enter critical) to [bend right] (semaphore);
\draw [->] (semaphore) to [bend right] (leave critical);
\draw [->] (critical) to (leave critical);
\draw [->] (leave critical) to [bend right] (waiting);
\end{tikzpicture}
\end{center}
\subsection{scope环境}
scope环境就是作用域控制,一个局域环境,参数只影响内部,外部的参数也影响不进来,不过值得一提的是,定义的点外面也可以用。scope环境一个有用的特性的里面的clip命令不会影响到外面。其用法:
\begin{verbatim}
\begin{scope}[options]
% only apply graphic options inside this scope, but not to anything outside.
\end{scope}
\end{verbatim}
\begin{lstlisting}
\begin{tikzpicture}[ultra thick]
\draw (0,0) -- (0,1);
\begin{scope}[thin]
\draw (1,0) -- (1,1);
\end{scope}
\draw (2,0) -- (2,1);
\end{tikzpicture}
\end{lstlisting}
\begin{center}
	\begin{tikzpicture}[ultra thick]
	\draw (0,0) -- (0,1);
	\begin{scope}[thin]
	\draw (1,0) -- (1,1);
	\end{scope}
	\draw (2,0) -- (2,1);
	\end{tikzpicture}
\end{center}
\subsection{迭代语句}
\begin{lstlisting}
\begin{tikzpicture}
\draw[help lines] (0,0) grid (3,2);
\foreach \x in {0,1,...,4}
\draw[xshift=\x cm] (0,-1) -- (0,1);
\end{tikzpicture}
\end{lstlisting}
\begin{center}
	\begin{tikzpicture}
	\draw[help lines] (0,0) grid (3,2);
	\foreach \x in {0,1,...,4}
	\draw[xshift=\x cm] (0,-1) -- (0,1);
	\end{tikzpicture}
\end{center}


其中... 表示一直这样有规律下去生成迭代列表。迭代语句有很多用法,详见后面的具体例子。
\subsection{其它}
\begin{enumerate}
	\item 平移:xshift ,x坐标轴平移。 yshift ,y坐标轴平移。 rotate ,旋转 。 注意xshift默认的单位并不是cm,如果要单位是cm需要写出来;
	\begin{center}
		\begin{tikzpicture} 
		\draw[help lines] (0,0) grid (3,2); 
		\draw (0,0) -- (1,1); 
		\draw[red] (0,0) -- ([xshift=1cm] 1,1); 
		\end{tikzpicture}
	\end{center}
	\item 旋转:后面加上可选项 rotate=30 即可,意思是图形逆时针旋转30度;
	\begin{center}
		\begin{tikzpicture} 
		\draw (0,0)[rotate=30]  ellipse (2 and 1); 
		\end{tikzpicture}
	\end{center}
	\item 反对称:xscale=-1或者yscale=-1就刚好相对y轴或x轴反对称;
	\item 翻转:例子如下
	\begin{lstlisting}
	\begin{tikzpicture}
	\draw[help lines, step=0.5] (0,0) grid (7,1.5);
	\draw[red, very thick] (0,0) -- (0,0.5) 
	[shift={(4pt,2pt)}] (0,0) -- (0,0.5);
	\draw[red, very thick] (0.5,0) -- (0.5,0.5) 
	[shift={+(4pt,2pt)}] (0.5,0) -- (0.5,0.5);
	\draw[rotate=30,fill=blue] (1.5,-1) rectangle (2,-0.5);
	\draw[rotate around={60:(3,0.5)},fill=blue] (2.5,0.25) rectangle (3,0.75);
	\draw[xscale=1,yscale=1.1,fill=green] (4,0.5) circle (0.5);
	\draw[xslant=2,very thick] (5,0) -- (5.5,0.5) -- (5.5,0);
	\end{tikzpicture}
	\end{lstlisting}
	\begin{center}
		\begin{tikzpicture}
		\draw[help lines, step=0.5] (0,0) grid (7,1.5);
		\draw[red, very thick] (0,0) -- (0,0.5) 
		[shift={(4pt,2pt)}] (0,0) -- (0,0.5);
		\draw[red, very thick] (0.5,0) -- (0.5,0.5) 
		[shift={+(4pt,2pt)}] (0.5,0) -- (0.5,0.5);
		\draw[rotate=30,fill=blue] (1.5,-1) rectangle (2,-0.5);
		\draw[rotate around={60:(3,0.5)},fill=blue] (2.5,0.25) rectangle (3,0.75);
		\draw[xscale=1,yscale=1.1,fill=green] (4,0.5) circle (0.5);
		\draw[xslant=2,very thick] (5,0) -- (5.5,0.5) -- (5.5,0);
		\end{tikzpicture}
	\end{center}
	\item 循环并行:其用法
	\begin{verbatim}
	\foreach \variable in {list of values}{
	\commands ;
	\end{verbatim}
	\begin{lstlisting}
	\begin{tikzpicture}
	\foreach \x in {-0.5cm,0cm,0.5cm}{
	\draw[red,very thick] (\x,-5pt) -- (\x,5pt);
	}
	\foreach \y in {-0.5cm,0cm,0.5cm}{
	\draw[blue,very thick] (1cm,\y) -- (1.5cm,\y);
	}
	\foreach \x in {0,...,9}{
	\draw[green,very thick] (\x,-1) circle (0.4cm);
	}
	\foreach \x in {2,2.5,...,4}{
	\draw[purple,very thick] (\x cm,-3pt) -- (\x cm,3pt);
	}
	\end{tikzpicture}
	\end{lstlisting}
	\begin{center}
	\begin{tikzpicture}
	\foreach \x in {-0.5cm,0cm,0.5cm}{
		\draw[red,very thick] (\x,-5pt) -- (\x,5pt);
	}
	\foreach \y in {-0.5cm,0cm,0.5cm}{
		\draw[blue,very thick] (1cm,\y) -- (1.5cm,\y);
	}
	\foreach \x in {0,...,9}{
		\draw[green,very thick] (\x,-1) circle (0.4cm);
	}
	\foreach \x in {2,2.5,...,4}{
		\draw[purple,very thick] (\x cm,-3pt) -- (\x cm,3pt);
	}
	\end{tikzpicture}
		\end{center}
	\begin{lstlisting}
	 \begin{tikzpicture}
	\foreach \x in {1,2,...,5,7,8,...,12}{
	\foreach \y in {1,...,5}{
	\draw (\x,\y) +(-0.5,-0.5) rectangle +(0.5,0.5);
	\draw (\x,\y) node{\x,\y};
	}
	}
	\end{tikzpicture}
	\end{lstlisting}
	\begin{center}
	 \begin{tikzpicture}
	\foreach \x in {1,2,...,5,7,8,...,12}{
		\foreach \y in {1,...,5}{
			\draw (\x,\y) +(-0.5,-0.5) rectangle +(0.5,0.5);
			\draw (\x,\y) node{\x,\y};
		}
	}
	\end{tikzpicture}
	\begin{lstlisting}
	\begin{tikzpicture}
	[L1Node/.style={circle,   draw=blue!50, fill=blue!20, very thick, minimum size=10mm},
	L2Node/.style={rectangle,draw=green!50,fill=green!20,very thick, minimum size=10mm}]
	\foreach \x in {1,...,5}
	\node[L1Node] (w1_\x) at (2*\x, 0){$\int_\Omega x_\x$};
	\end{tikzpicture}
	\end{lstlisting}
	\begin{center}
		\begin{tikzpicture}
		[L1Node/.style={circle,   draw=blue!50, fill=blue!20, very thick, minimum size=20mm},
		L2Node/.style={rectangle,draw=green!50,fill=green!20,very thick, minimum size=10mm}]
		\foreach \x in {1,...,5}
		\node[L1Node] (w1_\x) at (2*\x, 0){$\int_\Omega x_\x\mathrm{d}x$};
		\end{tikzpicture}
	\end{center}
	\end{center}
	\item 确定路径:
	\begin{itemize}
		\item 线条:path路径是最基本的命令,draw命令等价于 \lstinline{\path[draw]} ,fill命令等价于 \lstinline{\path[fill]} ,filldraw命令等价于 \lstinline{\path[draw,fill]} ,其他clip,shade命令情况类似。
		\item 虚线和点线:线条除了之前说的dashed和dotted两种样式之外,还有loosely dashed,densely dashed和loosely dotted, densely dotted;
		\item 线条的粗细:其他选项还有 ultra thin , very thin, thin, semithick, very thick, ultra thick
			\begin{center}
		\begin{tikzpicture} 
		\draw [ultra thick] (0,1) -- (2,1); 
		\draw [thick] (0,0.5) -- (2,0.5); 
		\draw [thin] (0,0) -- (2,0); 
		\end{tikzpicture}
			\end{center}
		或者直接通过可选项line width来定义:
			\begin{center}
		\begin{tikzpicture} 
		\draw [line width=0.4pt] (0,0) -- (2,0); 
		\draw[red]  (0,1) -- (2,1); 
		\draw [line width=0.2cm] (4,.75) -- (5,.25); 
		\end{tikzpicture}	
			\end{center}
	\end{itemize}
\end{enumerate}
\section{pgfplots宏包}
pgfplots宏包真是太好了,有时甚至画一个基本的坐标轴都懒得动用其他宏包命令了,我可以直接调用一个axis环境和进行一些简单的优化即可。当然就作为坐标轴作图可能总是用pgfplots宏包可能会稍显单调,但如果要求不是特别高的确实用pgfplots宏包会基于坐标轴的各个图形非常的称心如意,比如下面两个例子直接画函数与根据数据点来绘制:
\begin{lstlisting}
\begin{tikzpicture} 
\begin{axis} 
\addplot {x^2}; 
\end{axis} 
\end{tikzpicture}
\begin{tikzpicture}
\begin{axis} 
\addplot coordinates  
{(0,0) 
(1,1) 
(2,3) 
(3,9)}; 
\end{axis}
\end{tikzpicture}
\end{lstlisting}
\begin{tikzpicture} 
\begin{axis} 
\addplot {x^2}; 
\end{axis} 
\end{tikzpicture}
\begin{tikzpicture}
\begin{axis} 
\addplot coordinates  
{(0,0) 
	(1,1) 
	(2,3) 
	(3,9)}; 
\end{axis}
\end{tikzpicture}
\begin{lstlisting}
\begin{tikzpicture}
\begin{axis}
\addplot[color=red]{exp(x)};
\end{axis}
\end{tikzpicture}
%Here ends the furst plot
\hskip 5pt
%Here begins the 3d plot
\begin{tikzpicture}
\begin{axis}
\addplot3[
surf,
]
{exp(-x^2-y^2)*x};
\end{axis}
\end{tikzpicture}
\end{lstlisting}
\begin{tikzpicture}
\begin{axis}
\addplot[color=red]{exp(x)};
\end{axis}
\end{tikzpicture}
%Here ends the furst plot
\hskip 5pt
%Here begins the 3d plot
\begin{tikzpicture}
\begin{axis}
\addplot3[
surf,
]
{exp(-x^2-y^2)*x};
\end{axis}
\end{tikzpicture}
\begin{lstlisting}
\begin{tikzpicture}
\begin{axis}[
axis lines = left,
xlabel = $x$,
ylabel = {$f(x)$},
]
%Below the red parabola is defined
\addplot [
domain=-10:10, 
samples=100, 
color=red,
]
{x^2 - 2*x - 1};
\addlegendentry{$x^2 - 2x - 1$}
%Here the blue parabloa is defined
\addplot [
domain=-10:10, 
samples=100, 
color=blue,
]
{x^2 + 2*x + 1};
\addlegendentry{$x^2 + 2x + 1$}

\end{axis}
\end{tikzpicture}
\end{lstlisting}
\begin{center}
\begin{tikzpicture}
\begin{axis}[
axis lines = left,
xlabel = $x$,
ylabel = {$f(x)$},
]
%Below the red parabola is defined
\addplot [
domain=-10:10, 
samples=100, 
color=red,
]
{x^2 - 2*x - 1};
\addlegendentry{$x^2 - 2x - 1$}
%Here the blue parabloa is defined
\addplot [
domain=-10:10, 
samples=100, 
color=blue,
]
{x^2 + 2*x + 1};
\addlegendentry{$x^2 + 2x + 1$}

\end{axis}
\end{tikzpicture}
\end{center}
\begin{lstlisting}
\begin{tikzpicture}
\begin{axis}[
title={Temperature dependence of CuSO$_4\cdot$5H$_2$O solubility},
xlabel={Temperature [\textcelsius]},
ylabel={Solubility [g per 100 g water]},
xmin=0, xmax=100,
ymin=0, ymax=120,
xtick={0,20,40,60,80,100},
ytick={0,20,40,60,80,100,120},
legend pos=north west,
ymajorgrids=true,
grid style=dashed,
]

\addplot[
color=blue,
mark=square,
]
coordinates {
(0,23.1)(10,27.5)(20,32)(30,37.8)(40,44.6)(60,61.8)(80,83.8)(100,114)
};
\legend{CuSO$_4\cdot$5H$_2$O}

\end{axis}
\end{tikzpicture}
\end{lstlisting}
\begin{center}
\begin{tikzpicture}
\begin{axis}[
title={Temperature dependence of CuSO$_4\cdot$5H$_2$O solubility},
xlabel={Temperature [\textcelsius]},
ylabel={Solubility [g per 100 g water]},
xmin=0, xmax=100,
ymin=0, ymax=120,
xtick={0,20,40,60,80,100},
ytick={0,20,40,60,80,100,120},
legend pos=north west,
ymajorgrids=true,
grid style=dashed,
]

\addplot[
color=blue,
mark=square,
]
coordinates {
	(0,23.1)(10,27.5)(20,32)(30,37.8)(40,44.6)(60,61.8)(80,83.8)(100,114)
};
\legend{CuSO$_4\cdot$5H$_2$O}

\end{axis}
\end{tikzpicture}
\end{center}
\begin{lstlisting}
\begin{tikzpicture}
\begin{axis}[
x tick label style={
/pgf/number format/1000 sep=},
ylabel=Year,
enlargelimits=0.05,
legend style={at={(0.5,-0.1)},
anchor=north,legend columns=-1},
ybar interval=0.7,
]
\addplot 
coordinates {(2012,408184) (2011,408348)
(2010,414870) (2009,412156) (2008,415 838)};
\addplot 
coordinates {(2012,388950) (2011,393007) 
(2010,398449) (2009,395972) (2008,398866)};
\legend{Men,Women}
\end{axis}
\end{tikzpicture}
\begin{tikzpicture}
\begin{axis}[
title=Exmple using the mesh parameter,
hide axis,
colormap/cool,
]
\addplot3[
mesh,
samples=50,
domain=-8:8,
]
{sin(deg(sqrt(x^2+y^2)))/sqrt(x^2+y^2)};
\addlegendentry{$\frac{\sin(r)}{r}$}
\end{axis}
\end{tikzpicture}
\end{lstlisting}
\begin{tikzpicture}
\begin{axis}[
x tick label style={
	/pgf/number format/1000 sep=},
ylabel=Year,
enlargelimits=0.05,
legend style={at={(0.5,-0.1)},
	anchor=north,legend columns=-1},
ybar interval=0.7,
]
\addplot 
coordinates {(2012,408184) (2011,408348)
	(2010,414870) (2009,412156) (2008,415 838)};
\addplot 
coordinates {(2012,388950) (2011,393007) 
	(2010,398449) (2009,395972) (2008,398866)};
\legend{Men,Women}
\end{axis}
\end{tikzpicture}
\begin{tikzpicture}
\begin{axis}[
title=Exmple using the mesh parameter,
hide axis,
colormap/cool,
]
\addplot3[
mesh,
samples=50,
domain=-8:8,
]
{sin(deg(sqrt(x^2+y^2)))/sqrt(x^2+y^2)};
\addlegendentry{$\frac{\sin(r)}{r}$}
\end{axis}
\end{tikzpicture}

\begin{lstlisting}
\begin{tikzpicture}
\begin{axis}
\addplot3[
surf,
] 
coordinates {
(0,0,0) (0,1,0) (0,2,0)

(1,0,0) (1,1,0.6) (1,2,0.7)

(2,0,0) (2,1,0.7) (2,2,1.8)
};
\end{axis}
\end{tikzpicture}
\begin{tikzpicture}
\begin{axis}
[
view={60}{30},
]
\addplot3[
domain=0:5*pi,
samples = 60,
samples y=0,
]
({sin(deg(x))},
{cos(deg(x))},
{x});
\end{axis}
\end{tikzpicture}

\end{lstlisting}
\begin{tikzpicture}
\begin{axis}
\addplot3[
surf,
] 
coordinates {
	(0,0,0) (0,1,0) (0,2,0)
	
	(1,0,0) (1,1,0.6) (1,2,0.7)
	
	(2,0,0) (2,1,0.7) (2,2,1.8)
};
\end{axis}
\end{tikzpicture}
\begin{tikzpicture}
\begin{axis}
[
view={60}{30},
]
\addplot3[
domain=0:5*pi,
samples = 60,
samples y=0,
]
({sin(deg(x))},
{cos(deg(x))},
{x});
\end{axis}
\end{tikzpicture}

\section{电路图}
电路基本符号,具体详细看\href{http://texdoc.net/texmf-dist/doc/latex/circuitikz/circuitikzmanual.pdf}{CircuiTikZ}
\begin{enumerate}
	\item battery : 电池\tikz[circuit ee IEC]{\node[battery] {};}
	\item bulb : 灯泡 \tikz[circuit ee IEC]{\node[bulb] {};}
	\item make contact : 开关 \tikz[circuit ee IEC]{\node[make contact] {};}
	\item make contact : 开关另一种形式 额外选项{[set make contact graphic= var make contact IEC graphic]} \tikz[circuit ee IEC,set make contact graphic= var make contact IEC graphic]{\node[make contact] {};}
	\item resistor 电阻 (加上选项[ohm=20k]则上面写上电阻数值) \tikz[circuit ee IEC]{\node[resistor] {};}
	\item contact 电线交点 \tikz[circuit ee IEC]{\node[contact] {};}
	\item current direction to路径上加上电流方向(如果是[\textbf{current direction'}]则方向反向。) \tikz[circuit ee IEC]{\draw (0,0) to[current direction] (1,0);}
\end{enumerate}

连线问题:各个元器件之间的连线除了一般的 -- 连直线外,还可以通过 -| 或者 |- 来处理垂直拐线的问题,其中 -| 你可以理解为从第一个点先横着走再竖着走,而 |- 你可以理解为先从第一个点竖着走再横着走。

翻转问题:四个基本的选项[ point up ,point down, point left, point right],分别是朝上,朝下,朝左和朝右,其他复杂的角度的处理方法不是用rotate选项,而是在路径上加上上面的电路符号选项,这样那些元器件会自动跟随路径对齐的。
\section{小总结}
TikZ只是一个前端(frontend),画图功能通过调用底层PGF宏包完成。
\begin{enumerate}
	\item 简易的tikz环境:一个分号表示画图的结束;
	\begin{lstlisting}
	\tikz ...;
	\end{lstlisting}
	可以有多个画图语句
	\begin{lstlisting}
	\tikz{...;...;...;}
	\end{lstlisting}
	\item 整体缩放图形倍数:magniication=1为原始大小
	\begin{lstlisting}
	\begin{tikzpicture}[scale=<magnification>]
	......
	\end{tikzpicture}
	\end{lstlisting}
	\item 自定义一个图形,可方便重复使用:\lstinline|\def\<name>{<a path>}|
	\begin{lstlisting}
	%自定义一个正方形 
	\def\rectangle{-- ++(1,0) -- ++(0,1) -- ++(-1,0) -- cycle}
	\end{lstlisting}
	\item 自定义一组样式,方便重复使用,并可设置参数:\lstinline|<name>/.style={<attributes>}|
	\begin{lstlisting}
help line/.style={very thin, color=#1red!20!blue!20,rounded corners=2pt}
	\end{lstlisting}
	\item 最基本画图命令:\lstinline|\path|
	\begin{lstlisting}
	\draw=\path[draw], \draw[color=<color>]=\path[draw=<color>]
	\fill=\path[fill], \fill[color=<color>]=\path[fill=<color>]
	\filldraw=\path[fill, draw]
	\clip=\path[clip]
	\end{lstlisting}
	\item 定位:
	\begin{lstlisting}
	+(x,y) 在之前的画笔点的基础上偏移(x,y),但并不改变画笔点;
	++(x,y)同样是偏移(x,y),但把偏移之后的点做为新的画笔点。
	\end{lstlisting}
	\item scope环境:需要注意\lstinline|\clip|的作用范围是从其语句之后一直到当前所在scope的结束
	\begin{lstlisting}
	\begin{scope}[<sequence of atttibutes>]
	......
	\end{scope}
	\end{lstlisting}
	\item 几个典例:
	\begin{itemize}
		\item 球坐标(angle:radius)
		\item 弧 (point at the initial angle) arc(inital angle:terminal angle:radius)
		\item 圆 (center) circle (radius)
		\item 椭圆 (center) ellipse (x radius and y radius)
		\item 矩形 (point 1) rectangle (point 2)
		\item 交点 (intersection of <line1> and <line2>)
	\end{itemize}
	\item 可用选项:
	\begin{itemize}
		\item 箭头:->, ->>, <-, <<-, <->
		\item 旋转:rotate=<angle>
		\item 圆角:rounded corners=<x pt>
		\item 颜色:color=<color1!percentage1><!color2!percentage2>...
		\item 虚线:dashed, loosely dashed, densely dashed, dotted, loosely dotted, densely dotted
		\item 宽度:very thin, thin (正常宽度), thick, very thick
	\end{itemize}
	\item 缩放:
	\begin{lstlisting}
	[scale=<magnification>],[xscale=<magnification>],[yscale=<manification>].
	\end{lstlisting}
	其中若<magnification>取值为实数,其绝对值表示缩放的倍数;若<magnification>是负数,表示进行翻转;若xscale是负数则左右翻转,yscale是负数则上下翻转,scale是负数则同时翻转。
	\item 偏移:
	\begin{lstlisting}
	[shift=<point>], [shift=+<point>],[xshift=<d_x>], [yshift=<d_y>].
	\end{lstlisting}
	将后面的图加上一个矢量。其中[shift=+<point>]表示,此加上的矢量为前一个画笔点+<point>。
\end{enumerate}