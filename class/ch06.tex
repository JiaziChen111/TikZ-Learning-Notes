\chapter{流程图}
\begin{center}
\tikzset{
	every picture/.style={
		>=Stealth,
		node distance=5mm and 5mm
}}
\tikzstyle{SS}      %开始结束
=[rounded corners, draw]
\tikzstyle{IO}      %输入输出
=[trapezium, trapezium left angle=70, trapezium right angle=110, draw]
\tikzstyle{NR}      %内容
=[draw]
\tikzstyle{PD}      %判断
=[diamond, aspect=4, draw, inner sep=1.5pt]\hfill
\begin{tikzpicture}[every node/.style={minimum width=1.8cm},->]
\node[draw,rounded corners,minimum height=0.6cm,minimum width=1.3cm](start){开始};
\node[rectangle,draw, below=of start](in){$A=\frac12$};
\node[rectangle,draw, below=of in,minimum height=0.6cm](n1){$k=1$};
\node[PD, below=of n1,minimum height=0.6cm](pd){$k\leqslant 2$};
\node[rectangle,draw,below=of pd,minimum height=0.6cm](M1){};
\node[rectangle,draw,below=of M1,minimum height=0.6cm](M2){$k=k+1$};
\node[IO,right=of M1](M3){输出$A$};
\node[SS,below=of M3,minimum width=1.3cm](M4){结束};
\draw (pd)--node[right=-0.55cm]{是}(M1);
\draw (pd) -| node[right=-.6cm]{否} (M3);
\draw (M2.west)--++(-0.5,0) |- ($(n1.south)!0.5!(pd.north)$);
\path(start)edge(in)
(in)edge(n1)
(n1)edge(pd)
(pd)edge(M1)
(M1)edge(M2)
(M3)edge(M4);
\end{tikzpicture}
% 流程图定义基本形状
\tikzstyle{ellipse}=[draw, rectangle, minimum width=2.8em, rounded corners=6pt,line width=0.5pt]% minimum height=1.5em, fill=red!20
\tikzstyle{pxsbx}=[trapezium, trapezium left angle=75, trapezium right angle=105, minimum width=3em, text centered, draw = black, fill=white,line width=0.5pt]
\tikzstyle{lingxing}=[draw,diamond,shape aspect=3,inner sep = 0.4pt,thick,font=\itshape,line width=0.5pt]
%,minimum size=8mm
\begin{tikzpicture}[node distance=1.2cm]
%定义流程图具体形状
\node (start) [ellipse,inner sep=1.5pt] {开始};
\node (srn) [pxsbx, below of=start,node distance=1.0cm,inner sep=1.5pt] {输入};
\node (yizhi) [minimum height=0cm,draw, below of=srn,node distance=1.0cm,inner sep=2pt] {过程};
\node (tiaojian) [lingxing,draw, below of=yizhi,inner sep=1.5pt] {判断};
\node (shi) [minimum height=0cm,draw, below of=tiaojian,inner sep=2pt] {循环1};
\node (fuzhi1) [minimum height=0cm,draw, below of=shi,inner sep=2pt] {循环2};
\node (fuzhi2) [minimum height=0cm,draw, below of=fuzhi1,inner sep=2pt] {循环3};
\node (fou) [right of=tiaojian,xshift=1cm,coordinate]  {}; 
\node (shuchus) [below of=fou,draw,pxsbx,inner sep=1.5pt,node distance=1.0cm]  {输出$\!S$}; 
\node (end) [below of=shuchus,draw,ellipse,inner sep=1.5pt,node distance=1.0cm]  {结束}; 
%连接具体形状
\draw[arrows={-Stealth[scale=0.8]}](start) -- (srn);
\draw[arrows={-Stealth[scale=0.8]}](srn) -- (yizhi) ;
\path (yizhi) -- (tiaojian) coordinate[pos=0.5](yt);
\node [left of=yt,coordinate,node distance=2.3cm] (c1)  {}; 
\draw[arrows={-Stealth[scale=0.8]}](yizhi) -- (tiaojian);
\draw[arrows={-Stealth[scale=0.8]}](tiaojian) -- node[right=-0.5mm,blue]{是}(shi);
\draw[arrows={-Stealth[scale=0.8]}](shi) -- (fuzhi1);
\draw[arrows={-Stealth[scale=0.8]}](fuzhi1)--(fuzhi2);
\draw[arrows={Stealth[scale=0.8]-}](yt)--(c1);
\draw(c1)|- (fuzhi2);
\draw(tiaojian)-- node[midway,above=-1mm,blue]{否}(fou);
\draw[arrows={-Stealth[scale=0.8]}](fou)--(shuchus);
\draw[arrows={-Stealth[scale=0.8]}](shuchus)--(end);
\end{tikzpicture}
\end{center}

\setlength{\unitlength}{1mm}
\newcommand{\wrt}[1]{\makebox(0,0)[c]{#1}}
\newcommand{\lline}[1]{\line(-1,0){#1}}
\newcommand{\rline}[1]{\line(1,0){#1}}
%\newcommand{\Uline}[1]{\line(0,1){#1}}
\newcommand{\dline}[1]{\line(0,-1){#1}}
%\newcommand{\lvec}[1]{\vector(-1,0){#1}}
\newcommand{\rvec}[1]{\vector(1,0){#1}}
\newcommand{\uvec}[1]{\vector(0,1){#1}}
\newcommand{\dvec}[1]{\vector(0,-1){#1}}
\newsavebox{\condition}
\newsavebox{\process}
\newsavebox{\inputoutput}
\savebox{\process}(0,0){\thicklines
	\put(-10,-3){\framebox(20,6){}}
}
\savebox{\condition}(0,0){\thicklines
	\put(-10,0){\line(2,1){10}}
	\put(-10,0){\line(2,-1){10}}
	\put(10,0){\line(-2,1){10}}
	\put(10,0){\line(-2,-1){10}}
}
\savebox{\inputoutput}(0,0){\thicklines
	\put(-10.5,-3){\rline{18}}
	\put(-10.5,-3){\line(1,2){3}}
	\put(10.5,3){\lline{18}}
	\put(10.5,3){\line(-1,-2){3}}
}
{\vspace*{1cm}\hspace*{3cm}
	\begin{picture}(45,73)(20,-73)\thicklines
	\put(40,3){\oval(15,6)}\put(40,3){\wrt{开始}} \put(40,0){\dvec{5}}
	\put(40,-8){\usebox{\inputoutput}} \put(40,-8){\wrt{输入$m,\,n$}}
	\put(40,-11){\dvec{5}} \put(40,-19){\usebox{\process}}
	\put(40,-19){\wrt{$i=1$}} \put(40,-22){\dvec{8}}
	\put(40,-33){\usebox{\process}} \put(40,-33){\wrt{$a=m\times i$}}
	\put(40,-36){\dvec{5}} \put(40,-46){\usebox{\condition}}
	\put(40,-46){\wrt{$n$整除$a$?}} \put(40,-51){\dvec{5}}
	\put(41,-53){\makebox(0,0)[l]{是}}
	\put(40,-59){\usebox{\inputoutput}} \put(40,-59){\wrt{输出$a,\,i$}}
	\put(40,-62){\dvec{5}}
	\put(40,-70){\oval(15,6)}\put(40,-70){\wrt{结束}}
	\put(30,-46){\line(-1,0){15}} \put(28,-45){\makebox(0,0)[b]{否}}
	\put(15,-46){\uvec{10}} \put(15,-33){\usebox{\process}}
	\put(15,-33){\wrt{$i=i+1$}} \put(15,-30){\line(0,1){5}}
	\put(15,-25){\rvec{25}}
\end{picture}}
% 流程图定义基本形状
\tikzstyle{ellipse}=[draw, rectangle, minimum width=2.8em, rounded corners=6pt,line width=0.5pt]% minimum height=1.5em, fill=red!20
\tikzstyle{pxsbx}=[trapezium, trapezium left angle=75, trapezium right angle=105, minimum width=3em, text centered, draw = black, fill=white,line width=0.5pt]
\tikzstyle{lingxing}=[draw,diamond,shape aspect=3,inner sep = 0.4pt,thick,font=\itshape,line width=0.5pt]%,minimum size=8mm
\begin{tikzpicture}[node distance=1.2cm]
%定义流程图具体形状
\node (start) [ellipse,inner sep=1.5pt] {开始};
\node (srn) [pxsbx, below of=start,node distance=1.0cm,inner sep=1.5pt] {输入$\!n=0$};
\node (yizhi) [minimum height=0cm,draw, below of=srn,node distance=1.0cm,inner sep=2pt] {$A=3^n-2^n$};
\node (panduan) [lingxing,draw, below of=yizhi,inner sep=1.5pt,minimum width=5em,minimum height=1.5em,node distance=1cm] {\phantom{$N$}};
\node (hh) [minimum height=0.95em,minimum width=3em,draw, left of=yizhi,node distance=1.0cm,inner sep=2pt,xshift=-0.8cm,yshift=-1mm] {$$};

\node (shuchu) [below of=panduan,draw,pxsbx,inner sep=1.5pt,node distance=1.1cm]  {输出$\!n$}; 
\node (end) [below of=shuchu,draw,ellipse,inner sep=1.5pt,node distance=1.0cm]  {结束};
%连接具体形状
\draw[arrows={-Stealth[scale=0.8]}](start) -- (srn);
\draw[arrows={-Stealth[scale=0.8]}](srn) -- (yizhi) ;
\path (srn) -- (yizhi) coordinate[pos=0.5](yt);
\draw[arrows={-Stealth[scale=0.8]}](yizhi) -- (panduan) ;
\draw[arrows={-Stealth[scale=0.8]}](panduan) -- (shuchu) node[pos=0.45,left=-1mm]{否};
\draw[arrows={-Stealth[scale=0.8]}](panduan) -| (hh)node[left=-1.4em,below=2.4em]{是};
\draw[arrows={-Stealth[scale=0.8]}](hh) |- (yt) ;
\draw[arrows={-Stealth[scale=0.8]}](shuchu)--(end);
\end{tikzpicture}

%例1
\begin{tikzpicture}[->,arrows={-Stealth[scale=0.8]}] 
%放框 
\node[SS](start){开始}; 
\node[IO, below=of start](in){输入$N$}; 
\node[NR, below=of in](n1){$k=1$, $S=0$}; 
\node[NR, below=of n1](n2){$S=S+\dfrac1{k(k+1)}$}; 
\node[NR, right=of n2](n3){$k=k+1$}; 
\node[PD, below=of n2](pd){$k<N$}; 
\node[IO, below=of pd](out){输出$S$}; 
\node[SS, below=of out](stop){结束}; 
%连线 
\path 
(start) edge (in) 
(in) edge (n1) 
(n1) edge (n2) 
(n2) edge (pd) 
(pd) edge node[right]{否}(out) 
(out) edge (stop); 
\draw (pd.east)node[below]{是} -| (n3); 
\draw (n3) |- ($(n1.south)!0.5!(n2.north)$); 
\end{tikzpicture}
%例2
\begin{tikzpicture}[->] 
\node[SS](start){开始}; 
\node[NR, below=of start](n1){$a=1$}; 
\node[NR, below right=1mm and 5mm of n1](n2){$a=a^2+2$}; 
\node[PD, below=8mm of n1](pd){$a<10$?}; 
\node[IO, below=of pd](out){输出$a$}; 
\node[SS, below=of out](stop){结束}; 
\path (start) edge (n1) 
(n1) edge (pd) 
(pd) edge node[right]{否}(out) 
(out) edge (stop); 
\draw (pd.east)node[below]{是} -| (n2); 
\draw (n2) -- (n2-|n1); 
\end{tikzpicture}
%例3
\begin{tikzpicture}[->] 
\node[SS](start){开始}; 
\node[IO, below=of start](in){输入$x$, $t$}; 
\node[NR, below=of in](n1){$M=1$, $S=3$, $k=1$}; 
\node[PD, below=of n1](pd){$k\le t$}; 
\node[NR, on grid, below left=1cm and 2cm of pd](n3){$M=\dfrac Mkx$}; 
\node[IO, on grid, below right=1cm and 2cm of pd](out){输出$S$}; 
\node[NR, below=of n3](n4){$S=M+S$}; 
\node[NR, below=of n4](n5){$k=k+1$}; 
\node[SS, below=of out](stop){结束}; 
\path (start) edge (in) 
(in) edge (n1) 
(n1) edge (pd) 
(n3) edge (n4) 
(n4) edge (n5) 
(out) edge (stop); 
\draw (pd) -| node[left]{是} (n3); 
\draw (pd) -| node[right]{否} (out); 
\draw (n5.west)--++(-0.5,0) |- ($(n1.south)!0.5!(pd.north)$); 
\end{tikzpicture}
