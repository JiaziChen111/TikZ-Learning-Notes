\chapter*{总结}
总体感觉, Tikz 画图操作比较精确, 重要的是它与\LaTeX 完美内嵌。 唯一一点不习惯的是它的代码看起来没有 Matlab 整齐。

有一个经疑问, 既然 TikZ 画图这么好看, 能否把它画的图另存为单独的文件呢?方法当然有, 就是使用 standalone 文档类。如下图就是用 standalone画好之后的 pdf 文件插入。

\begin{tikzpicture}[>= stealth]
\draw [->] (0,0,0) -- (2,0,0) node [at end, right] {$x$ 轴};
\draw [->] (0,0,0) -- (0,2,0) node [at end, left] {$y$ 轴};
\draw [->] (0,0,0) -- (0,0,2) node [at end, left] {$z$ 轴};
\filldraw [red, rotate around z=0] (0,0) -- (1,1) -- (1,0);
\draw [green, rotate around z=45] (0,0) -- (1,1) -- (1,0);
\draw [blue, rotate around z=90] (0,0) -- (1,1) -- (1,0);
\end{tikzpicture}
\begin{tikzpicture}[>=Stealth,scale=0.2,smooth]
%\draw[help lines] (0,0) grid (3,2);
\draw[->](-10,0)--(0,0) node[below left=-1.2pt]{$O$}--(20,0) node[below=1.5pt]{$x$};
\draw[->](0,-8)--(0,10)node[left]{$y$};
%\draw [circle] (0,0)  circle (5) ;
\coordinate (a) at (15,10);
\node [circle,draw] (P) at (5,0) [minimum size=50pt,below] {$P$};
\draw[blue] (a) -- (tangent cs:node=P,point={(a)},solution=1)
-- (P.center) -- (tangent cs:node=P,point={(a)},solution=2)-- cycle;
\end{tikzpicture}
\begin{center}
\begin{tikzpicture}[
every node/.style ={inner sep=0pt},
pgfornamentstyle/.style={color=green!50!black,fill=green!80!black}]
\node[text width=12cm,outer sep=1.2cm,text centered,color=red!90! black](Greeting)
{\calligra \Huge 第一届八一赛\\ 网络大学生数学竞赛\\2019年8月1号上午9点 \\
	\pgfornament[color=red!90!black,width=4cm]{72}};
\foreach \corner/\sym in {north west/none, north east/v, south west/h, south east/c}{
	\node[anchor=\corner](\corner)
	at(Greeting.\corner)
	{\pgfornament[width=2cm,symmetry=\sym]{63}};}
\path(north west)--(south west)
node[midway, anchor=east]
{\pgfornament[height=2cm]{9}}
(north east)--(south east)
node[midway,anchor=west]
{\pgfornament[height=2cm, symmetry=v]{9}};
\pgfornamenthline{north west}{north east}{north}{87}
\pgfornamenthline{south west}{south east}{south}{87}
\end{tikzpicture}
\end{center}