\usepackage{amsmath}
\usepackage{url}
\usepackage{xcolor}
\usepackage{color}
\usepackage{cancel}
\usepackage{pgfornament}
\usepackage{calligra}
\usepackage{pstricks-add}
%\usepackage{tkz-linknodes}
\usepackage{metalogo}
\usepackage{scalerel} %\scaleobj{1.5}{} 缩放公式大小
\usepackage{siunitx} %需用到\ang[options]{degrees}命令
\definecolor{pink}{RGB}{249,164,186}
\definecolor{grassgreen}{RGB}{128,255,0}
\def\sgn{\mathop{\rm sgn}}
%\setCJKmainfont[AutoFakeBold,AutoFakeSlant]{STSong}
%\setCJKsansfont[AutoFakeBold,AutoFakeSlant]{KaiTi}
%\setCJKmonofont[AutoFakeBold,AutoFakeSlant]{STFangsong}
%\usepackage{verbatim}
%\usepackage[active,tightpage]{preview}
%\PreviewEnvironment{tikzpicture}
%\setlength\PreviewBorder{5pt}
\usepackage{tikz,pgfplots}
\usepackage{circuitikz}
\usetikzlibrary{
	shapes.geometric,   %几何形状
	calc,               %计算
	positioning,       %用于提供位置表达语句的支持,通俗讲就是定位
	arrows.meta,        %提供各种箭头
	automata,          %绘制状态转换图
	matrix,            %构建矩阵
	topaths,%用于to/edge算子构建的路径
	shapes.misc,       %约分
	chains,          %绘制链状图形的库,一般画流程图必需的库
	shadings,          %提供已定义好的图案如菱形
	quotes,    %提供引用句法
	angles,   %用来标示角
	patterns,  %定义图样,如条纹
	fadings,   %用于灰度图
	fit,   %创建数个坐标点
	decorations.pathmorphing,   %装饰线
	decorations.text,
	quotes,through,
	intersections, %两个路径的交点
	backgrounds,   %背景路径
	shapes,
	arrows,    %箭头路径
	snakes,
	mindmap, %思维导图
	shadows,
	circuits.ee.IEC%电路图
}
\tikzstyle{state}=[circle,label=below:$r$,
thick,
minimum size=1.2cm,
draw=blue!80,
fill=blue!20]
\tikzstyle{state2}=[circle,label=below:$e$,
thick,
minimum size=1.2cm,
draw=blue!80,
fill=blue!20]
\tikzstyle{state3}=[circle,label=below:$a$,
thick,
minimum size=1.2cm,
draw=blue!80,
fill=blue!20]
\tikzstyle{state4}=[circle,label=below:$d$,
thick,
minimum size=1.2cm,
draw=blue!80,
fill=blue!20]
% The measurement vector is represented by an orange circle.
\tikzstyle{measurement}=[circle,
thick,
minimum size=1.2cm,
draw=orange!80,
fill=orange!25]

% The control input vector is represented by a purple circle.
\tikzstyle{input}=[circle,
thick,
minimum size=1.2cm,
draw=purple!80,
fill=purple!20]

% The input, state transition, and measurement matrices
% are represented by gray squares.
% They have a smaller minimal size for aesthetic reasons.
\tikzstyle{matrx}=[rectangle,
thick,
minimum size=1cm,
draw=gray!80,
fill=gray!20]

% The system and measurement noise are represented by yellow
% circles with a "noisy" uneven circumference.
% This requires the TikZ library "decorations.pathmorphing".
\tikzstyle{noise}=[circle,
thick,
minimum size=1.2cm,
draw=yellow!85!black,
fill=yellow!40,
decorate,
decoration={random steps,
	segment length=2pt,
	amplitude=2pt}]

% Everything is drawn on underlying gray rectangles with
% rounded corners.
\tikzstyle{background}=[rectangle,
fill=gray!10,
inner sep=0.2cm,
rounded corners=5mm]





\usepackage{scalerel} %\scaleobj{1.5}{} 缩放公式大小
\newcommand\hcancel[2][black]{\setbox0=\hbox{#2}\rlap{\raisebox{.45\ht0}{\textcolor{#1}{\rule{\wd0}{1pt}}}}#2}


\definecolor{pink}{RGB}{249,164,186}
\definecolor{grassgreen}{RGB}{128,255,0}

\pgfkeys{/pgf/decoration/.cd,
	stipple density/.store in=\pgfstippledensity,
	stipple density=.1,
	stipple scaling function/.store in=\pgfstipplescalingfunction,
	stipple scaling function=sin(\pgfstipplex*180)*0.875+0.125,
	stipple radius/.store in=\pgfstippleradius,
	stipple radius=0.25pt
}
\pgfdeclaredecoration{stipple}{draw}{
	\state{draw}[width=\pgfdecorationsegmentlength]{
		\pgfmathparse{\pgfdecoratedcompleteddistance/\pgfdecoratedpathlength}%
		\let\pgfstipplex=\pgfmathresult%
		\pgfmathparse{int(\pgfstippledensity*100)}%
		\let\pgfstipplen=\pgfmathresult
		\pgfmathloop
		\ifnum\pgfmathcounter<\pgfmathresult\relax
		\pgfpathcircle{%
			\pgfpoint{(rnd)*\pgfdecorationsegmentlength}%
			{(\pgfstipplescalingfunction)*(rnd^4)*\pgfdecorationsegmentamplitude+\pgfstippleradius}}% 
		{\pgfstippleradius}%
		\repeatpgfmathloop
	}
}

\tikzset{stipple/.style={
		decoration={stipple, segment length=2pt, #1},
		decorate,
		fill
}}
% end of stippling code

\newcommand\DrawBlock[3]{
	\ifx#1b\relax
	\path[draw]
	(lm\the\numexpr#2-1\relax) -- ++(0,0,#3) coordinate (blocklf)
	(bm\the\numexpr#2-1\relax) -- ++(0,0,#3) coordinate (blocklb)
	(lm#2) -- ++(0,0,#3) coordinate (blockrf)
	(bm#2) -- ++(0,0,#3) coordinate (blockrb);
	\filldraw[fill=white,draw=black]
	(lm\the\numexpr#2-1\relax) -- (blocklf) -- (blocklb) -- (blockrb) -- (blockrf) -- (lm#2);
	\else  
	\ifx#1f\relax
	\path[draw]
	(fm\the\numexpr#2-1\relax) -- ++(0,0,#3) coordinate (blocklf)
	(lm\the\numexpr#2-1\relax) -- ++(0,0,#3) coordinate (blocklb)
	(fm#2) -- ++(0,0,#3) coordinate (blockrf)
	(lm#2) -- ++(0,0,#3) coordinate (blockrb);
	\filldraw[fill=white,draw=black]
	(fm\the\numexpr#2-1\relax) -- (blocklf) -- (blocklb) -- (blockrb) -- (blockrf) -- (fm#2);
	\fi
	\fi
	\draw (blocklf) -- (blockrf);
}

\tikzset{
	every picture/.style={
		>=latex,
		node distance=5mm and 5mm
		%有无and的区别见手册positioning部分
}}

%定义框形状
\tikzstyle{SS}      %开始结束
=[rounded corners, draw,rectangle]
\tikzstyle{IO}      %输入输出
=[trapezium, trapezium left angle=70, trapezium right angle=110, draw]
\tikzstyle{NR}      %内容
=[draw]
\tikzstyle{PD}      %判断
=[diamond, aspect=4, draw, inner sep=1.5pt]



%\usepackage{background}
%\definecolor{bg}{RGB}{58,95,205}
%\backgroundsetup{opacity=0.05,scale=1.2,angle=0,contents={
%		\begin{tikzpicture}
%		\node[minimum width=18cm,minimum height=25cm,inner sep=0pt,outer sep=0pt,](box){};
%		\fill[left color=bg!70!black!70,right color=bg!70!black] (box.south west)--(box.south east)--(box.north east)--(box.north west)--cycle;
%		\shade[inner color=bg!70!black,outer color=white!30!bg] (3,6)circle(.3cm);
%		\draw[opacity=.2,blue] (4,-6)circle(3cm);
%		\draw[opacity=.2,blue] (3,-5)circle(5cm);
%		\draw[opacity=.2,blue] (2,-3)circle(6cm);
%		\shade[inner color=bg!70!black,outer color=white!30!bg,xshift=-2cm,yshift=-11cm] (3,6)circle(.3cm);
%		\shade[inner color=bg!70!black,outer color=white!30!bg,xshift=1.3cm,yshift=-12cm] (3,6)circle(.3cm);
%		\shade[inner color=bg!70!black,outer color=white!30!bg,xshift=-4cm,yshift=-8cm] (3,6)circle(.3cm);
%		\shade[inner color=bg!70!black,outer color=white!30!bg,xshift=-6cm,yshift=-10cm] (3,6)circle(.3cm);
%		\shade[inner color=bg!70!black,outer color=white!30!bg,xshift=-6cm,yshift=-13cm] (3,6)circle(.3cm);
%		\end{tikzpicture}
%}}

%\usepackage{draftwatermark}
%\SetWatermarkText{\shortstack
%	{微信公众号:八一考研数学竞赛
%}}
%\SetWatermarkLightness{0.50}%设置水印亮度
%\SetWatermarkScale{0.3}%设置水印大小